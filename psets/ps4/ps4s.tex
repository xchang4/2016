%% Please fill in your name and collaboration statement here.
\newcommand{\studentName}{**FILL IN YOUR NAME HERE**}
\newcommand{\collaborationStatement}{**FILL IN YOUR COLLABORATION STATEMENT HERE \\ (See the syllabus for information)**}


%%%%%%%%%%%%%%%%%%%%%%%%%%%%%%%%%%%%%%%%%%%%%%%
\documentclass[solution, letterpaper]{cs20}
\usepackage{enumerate}
\usepackage{tikz}
\usepackage{pgf}
\usepackage{tikz}
\usepackage{hyperref}
\begin{document}
\header{1}{Due Wednesday, March 3, 2016 at 9:59am. All students should submit an electronic copy.}


%%%%%%%%%%%%%%%%%%%%%%%%%%%%%%%%%%%%%%%%%%%%%%%
\PART{Hannah}
%%%%%%%%%%%%%%%%%%%%%%%%%%%%%%%%%%%%%%%%%%%%%%%
\problem{2+2}{1/3 page}
In this problem, we will prove that for any two sets $X$ and $Y$, $\overline{X \cup Y} = \overline{X} \cap \overline{Y}$. We will prove $\overline{X \cup Y} \subseteq \overline{X} \cap \overline{Y}$ in part (A) and $\overline{X} \cap \overline{Y} \subseteq \overline{X \cup Y}$ in part (B).

\subproblem Consider an element $a \in \overline{X \cup Y}$ and prove $a \in \overline{X} \cap \overline{Y}$.
\subproblem Consider an element $b \in \overline{X} \cap \overline{Y}$ and prove $b \in \overline{X \cup Y}$.

\begin{solution}
\subsolution Consider an element $a \in \overline{X \cup Y}$. 

By the definition of complement, $\neg (a \in X \cup Y)$. 

By the definition of union, $\neg (a \in X \vee a \in Y)$. 

Using De Morgan's Law, $\neg (a \in X) \wedge \neg (a \in Y)$. 

By the definition of complement, $a \in \overline{X} \wedge a \in \overline{Y}$.

By the definition of intersection, $a \in \overline{X} \cap \overline{Y}$.
\subsolution Consider an element $b \in \overline{X} \cap \overline{Y}$.

By the definition of intersection, $b \in \overline{X} \wedge b \in \overline{Y}$.

By the definition of complement, $\neg (b \in X) \wedge \neg (b \in Y)$.

Using De Morgan's Law, $\neg (b \in X \vee b \in Y)$.

By the definition of union, $\neg (b \in X \cup Y)$.

By the definition of complement, $b \in \overline{X \cup Y}$.

\end{solution}

\problem{2+2+2}{1/3 page}
Let $A$ and $B$ be finite sets such that $|A| = n$ and $|B| = m$. What are the minimum and maximum sizes of each of the following sets? (Justify each answer in 1-2 sentences. No formal proof needed.)

\subproblem $A \cup B$
\subproblem $A \cap B$
\subproblem $A - B$
 
\begin{solution}
\subsolution If $A$ and $B$ have no elements in common, then their union contains all of the elements of $A$ and of $B$ with no overlap. The maximum size is therefore $n + m$. 

\noindent If the smaller of $A$ and $B$ is a subset of the larger, then their union is equal to the larger set. The minimum size is therefore $max(n, m)$.
\subsolution If the smaller of $A$ and $B$ is a subset of the larger, then their intersection is equal to the smaller set. The maximum size is therefore $min(n, m)$.

\noindent If $A$ and $B$ have no elements in common, then their intersection is the empty set. The minimum size is therefore 0.
\subsolution If $A$ and $B$ have no elements in common, then $A - B$ is simply equal to $A$ (since no elements are removed). The maximum size is therefore $n$.

\noindent If $A$ is a subset of $B$, then $A - B$ is the empty set (since $A - B$ consists of all elements of $A$which are not in $B$, and all of the elements of $A$ are in $B$). The minimum size is therefore 0.
\end{solution}

%%%%%%%%%%%%%%%%%%%%%%%%%%%%%%%%%%%%%%%%%%%%%%%
\PART{Erin}
%%%%%%%%%%%%%%%%%%%%%%%%%%%%%%%%%%%%%%%%%%%%%%%
\problem{4}{1/3 page}
Find a bijection between the set of all integers and the set of all natural numbers (including 0). 

\begin{solution}
We define the function $f:\mathbb{N} \to \mathbb{Z}$ as follows: 
$$f(n) = \begin{cases} \frac{n+1}{2} & \mbox{if }n\mbox{ is odd} \\ -\frac{n}{2} & \mbox{if }n\mbox{ is even} \end{cases}$$
and its inverse $g:\mathbb{Z} \to \mathbb{N}$
$$g(z) = \begin{cases} -2z & \mbox{if }z \leq 0 \\ 2z-1& \mbox{if }z > 0 \end{cases}$$
Since we are able to define functions between the two sets in both directions, we have a bijection, and thus the two sets have the same cardinality.
\end{solution}
 
 
\problem{3+3}{1/2 page}
In class, you showed that if $A$ and $B$ are finite sets with the same cardinality and $f : A \rightarrow B$ is a total injective function, then $f$ is also surjective. 
\subproblem 
Now, prove that if $f$ is a total surjective function, then $f$ is also injective (thus, $f$ is surjective if and only if it is injective).
\subproblem
Finally, prove that in this case, there exists an inverse function $g : B \rightarrow A$ such that $g \circ f$ is the identity function.

\begin{solution}
\subsolution
\textit{Proof by Contradiction:}\\
Suppose that $f$ is not injective. Then, there exists some element in $B$ (let's call it $b^*$) which is mapped to by 2 elements in $A$ (let's call them $a_1$ and $a_2$). \\ \\
\noindent Let's look at the set $C = A - \{a_1, a_2\}$. Because $f$ is a total function and $C \subset A$, we know that for all elements $c \in C$, there exists a $b$ in $B$ such that $f(c) = b$. Let's say that $|A| = n$ so $|C| = n - 2$. \\ \\
\noindent Collecting at all of these $b$ values, we put them in a set $D$, where we know that $D \leq n - 2$. \\ \\
\noindent Because $a_1$ and $a_2$ both map to $b^* = f(a_1) = f(a_2) \in B$, we can construct the set $E = \{f(x) : \forall x \in A\} = D \cup \{b^*\}$ where $|E| = |D| + |\{b^*\}| \leq n - 1$. Because $|E| \leq n-1$, we know that $E \subset B$.  \\ \\
\noindent Thus, we know there exists some $b_0$ which is in $B$ but not in $E$, which means that $f$ doesn't map any element of $A$ to $b_0 \in B$, and so $f$ is not surjective. \\ \\ \\ \\

\subsolution
\textit{Proof by Construction:}\\
Let's construct the function $g: B \to A$ such that $g \circ f$ is the identity function. \\ \\
\noindent In this case, $f$ is surjective and injective; thus, $f$ is bijective. Because $f$ is a bijective function, for every element $b \in B$, there exists exactly 1 element $a \in A$ where $f(a) = b$. This is true for every element in $A$ since $f$ is a total function. \\ \\
\noindent Thus, let's define $g$ as follows:
$$\forall y \in B, g(y) = x\text{, where }x \in A \wedge f(x) = y$$
(That is, define $g$ as the inverse function of $f$ : $g = f^{-1}(y)$.)\\\\
This is a valid definition because $f$ is surjective. Furthermore, the above definition is a total function because it's defined for all $y \in B$; $g$ is also a bijective function because it's defined as the inverse function of $f$, which is bijective.\\\\
Then, $g \circ f: A \to A$ is the identity function where
$$(g \circ f)(x) = g(f(x)) = g(y) = x$$
where $\forall x \in A, y = f(x)$.

\end{solution}

\end{document}
