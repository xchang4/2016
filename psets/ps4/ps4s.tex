%% Please fill in your name and collaboration statement here.
\newcommand{\studentName}{**FILL IN YOUR NAME HERE**}
\newcommand{\collaborationStatement}{**FILL IN YOUR COLLABORATION STATEMENT HERE \\ (See the syllabus for information)**}


%%%%%%%%%%%%%%%%%%%%%%%%%%%%%%%%%%%%%%%%%%%%%%%
\documentclass[solution, letterpaper]{cs20}
\usepackage{enumerate}
\usepackage{tikz}
\usepackage{pgf}
\usepackage{tikz}
\usepackage{hyperref}
\begin{document}
\header{1}{Due Wednesday, March 3, 2016 at 9:59am. All students should submit an electronic copy.}


%%%%%%%%%%%%%%%%%%%%%%%%%%%%%%%%%%%%%%%%%%%%%%%
\PART{Hannah}
%%%%%%%%%%%%%%%%%%%%%%%%%%%%%%%%%%%%%%%%%%%%%%%
\problem{2+2}{1/3 page}
In this problem, we will prove that for any two sets $X$ and $Y$, $\overline{X \cup Y} = \overline{X} \cap \overline{Y}$. We will prove $\overline{X \cup Y} \subseteq \overline{X} \cap \overline{Y}$ in part (A) and $\overline{X} \cap \overline{Y} \subseteq \overline{X \cup Y}$ in part (B).

\subproblem Consider an element $a \in \overline{X \cup Y}$ and prove $a \in \overline{X} \cap \overline{Y}$.
\subproblem Consider an element $b \in \overline{X} \cap \overline{Y}$ and prove $b \in \overline{X \cup Y}$.

\begin{solution}
\subsolution Consider an element $a \in \overline{X \cup Y}$. 

By the definition of complement, $\neg (a \in X \cup Y)$. 

By the definition of union, $\neg (a \in X \vee a \in Y)$. 

Using De Morgan's Law, $\neg (a \in X) \wedge \neg (a \in Y)$. 

By the definition of complement, $a \in \overline{X} \wedge a \in \overline{Y}$.

By the definition of intersection, $a \in \overline{X} \cap \overline{Y}$.
\subsolution Consider an element $b \in \overline{X} \cap \overline{Y}$.

By the definition of intersection, $b \in \overline{X} \wedge b \in \overline{Y}$.

By the definition of complement, $\neg (b \in X) \wedge \neg (b \in Y)$.

Using De Morgan's Law, $\neg (b \in X \vee b \in Y)$.

By the definition of union, $\neg (b \in X \cup Y)$.

By the definition of complement, $b \in \overline{X \cup Y}$.

\end{solution}

\problem{2+2+2}{1/3 page}
Let $A$ and $B$ be finite sets such that $|A| = n$ and $|B| = m$. What are the minimum and maximum sizes of each of the following sets? (Justify each answer in 1-2 sentences. No formal proof needed.)

\subproblem $A \cup B$
\subproblem $A \cap B$
\subproblem $A - B$
 
\begin{solution}
\subsolution If $A$ and $B$ have no elements in common, then their union contains all of the elements of $A$ and of $B$ with no overlap. The maximum size is therefore $n + m$. 

\noindent If the smaller of $A$ and $B$ is a subset of the larger, then their union is equal to the larger set. The minimum size is therefore $max(n, m)$.
\subsolution If the smaller of $A$ and $B$ is a subset of the larger, then their intersection is equal to the smaller set. The maximum size is therefore $min(n, m)$.

\noindent If $A$ and $B$ have no elements in common, then their intersection is the empty set. The minimum size is therefore 0.
\subsolution If $A$ and $B$ have no elements in common, then $A - B$ is simply equal to $A$ (since no elements are removed). The maximum size is therefore $n$.

\noindent If $A$ is a subset of $B$, then $A - B$ is the empty set (since $A - B$ consists of all elements of $A$which are not in $B$, and all of the elements of $A$ are in $B$). The minimum size is therefore 0.
\end{solution}

%%%%%%%%%%%%%%%%%%%%%%%%%%%%%%%%%%%%%%%%%%%%%%%
\PART{Erin}
%%%%%%%%%%%%%%%%%%%%%%%%%%%%%%%%%%%%%%%%%%%%%%%
\problem{4}{1/3 page}
Find a bijection between the set of all integers and the set of all natural numbers (including 0). Be sure to explain why your relation is bijective.

\begin{solution}
We define the function $f:\mathbb{N} \to \mathbb{Z}$ as follows: 
$$f(n) = \begin{cases} \frac{n+1}{2} & \mbox{if }n\mbox{ is odd} \\ -\frac{n}{2} & \mbox{if }n\mbox{ is even} \end{cases}$$
and its inverse $g:\mathbb{Z} \to \mathbb{N}$
$$g(z) = \begin{cases} -2z & \mbox{if }z \leq 0 \\ 2z-1& \mbox{if }z > 0 \end{cases}$$
Since we are able to define functions between the two sets in both directions, we have a bijection, and thus the two sets have the same cardinality.
\end{solution}
 
 
\problem{4}{1/4 page}
If you have a bijective function $f(x)$ and it?s inverse $g(x) = f^{-1}(x)$, prove that your function $g(x)$ is also bijective.

\begin{solution}
\noindent We define $g$ as follows:
$$\forall y \in B, g(y) = x\text{, where }x \in A \wedge f(x) = y$$
\noindent Because $f$ is bijective, meaning that every element of $A$ maps to exactly one element of $B$, $g$ is also bijective, because each element of $B$ maps directly back to exactly one element of $A$.

\end{solution}

\end{document}
