%% Please fill in your name and collaboration statement here.
\newcommand{\studentName}{**FILL IN YOUR NAME HERE**}
\newcommand{\collaborationStatement}{**FILL IN YOUR COLLABORATION STATEMENT HERE \\ (See the syllabus for information)**}


%%%%%%%%%%%%%%%%%%%%%%%%%%%%%%%%%%%%%%%%%%%%%%%
\documentclass[solution, letterpaper]{cs20}
\usepackage{enumerate}
\usepackage{tikz}
\usepackage{pgf}
\usepackage{tikz}
\usepackage{hyperref}
\begin{document}
\header{5}{Due Wednesday, March 9, 2016 at 9:59am. All students should submit an electronic copy.}

%%%%%%%%%%%%%%%%%%%%%%%%%%%%%%%%%%%%%%%%%%%%%%%
\PART{Ben}
%%%%%%%%%%%%%%%%%%%%%%%%%%%%%%%%%%%%%%%%%%%%%%%
\problem{4}{1/3 page}
Show by giving an example for each case that the intersection of two uncountable sets can be: empty, finite, countably
infinite, or uncountably infinite.

\problem{4+2}{1 page}
In the large city of Hundred-land, there is a house for every real number between $1$ and $100$, inclusive. Let's assume for the sake of the problem that every house in Hundred-land is occupied by at least one citizen. In addition, every house is labelled with a unique real number between $1$ and $100$.

\subproblem The Schr\"{o}der-Bernstein Theorem states that for sets $S$ and $T$, if there exist injective functions $f : S \rightarrow T$ and $g : T \rightarrow S$, then $S$ and $T$ have the same cardinality. The neighboring city of Hundred-tropolis has a house for every real number between, but not including, $1$ and $100$. Using the Schr\"{o}der-Bernstein Theorem, show that Hundred-land and Hundred-tropolis have the same number of houses.

\subproblem The band One Direction pays a visit to Hundred-land as part of their world tour and, absolutely enamored, the mayor declares the city to now be named One-land. As such, each house will now have to be labelled with a unique real number from $0$ to $1$, inclusive. Is it possible to accomplish this? Devise a system so that each citizen in Hundred-land can re-label their house or, if such a labelling system does not exist, explain why.

\problem{2+2}{1/2 page}
A robot named Wall-E wanders around a two-dimensional grid. He starts at (0,0) and is allowed to take four different types of steps:
\begin{enumerate}
\item (-2, +2)
\item (-4, +4)
\item (+1, -1)
\item (+3, -3)
\end{enumerate}
For example, Wall-E might take the following stroll. The types of his steps are denoted by each arrow's subscript:
$$(0,0) \to_1 (-2,2) \to_3 (-1,1) \to_2 (-5,5) \to_4 (-2,2) \to \ldots$$
Wall-E's true love, the fashionable and high-powered robot, Eve, awaits at (1, 2).
\subproblem Describe a state machine model of this problem.
\subproblem Will Wall-E ever find his true love? Either find a path from Wall-E to Eve or use the Invariant Principle to prove that no such path exists.

%%%%%%%%%%%%%%%%%%%%%%%%%%%%%%%%%%%%%%%%%%%%%%%
\PART{Crystal}
%%%%%%%%%%%%%%%%%%%%%%%%%%%%%%%%%%%%%%%%%%%%%%%
\problem{3+1}{1/2 page}
\subproblem Give a recursive definition of the set $S$ of bit strings with no more than a single 1 in them 

(e.g. 00010, 010, or 000)
\subproblem Is $0010\in S$? How can you derive it from your base case?

\problem{2+2}{1/2 page}
Let $A = \left\{ {5n|n\in\mathbb{N}}\right\}$ and let $S$ be the set defined as follows:
\begin{itemize}
\item Base Case: $5 \in S$
\item Constructor Rule: If $x\in S$ and $y\in S$, $(x+y) \in S$ 
\end{itemize}
\subproblem Use induction to prove that $A\subseteq S $.
\subproblem Use structural induction to prove that $S\subseteq A$.

\end{document}
