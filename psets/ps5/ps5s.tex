%% Please fill in your name and collaboration statement here.
\newcommand{\studentName}{**FILL IN YOUR NAME HERE**}
\newcommand{\collaborationStatement}{**FILL IN YOUR COLLABORATION STATEMENT HERE \\ (See the syllabus for information)**}


%%%%%%%%%%%%%%%%%%%%%%%%%%%%%%%%%%%%%%%%%%%%%%%
\documentclass[solution, letterpaper]{cs20}
\usepackage{enumerate}
\usepackage{tikz}
\usepackage{pgf}
\usepackage{tikz}
\usepackage{hyperref}
\begin{document}
\header{5}{Due Wednesday, March 9, 2016 at 9:59am. All students should submit an electronic copy.}

%%%%%%%%%%%%%%%%%%%%%%%%%%%%%%%%%%%%%%%%%%%%%%%
\PART{Ben}
%%%%%%%%%%%%%%%%%%%%%%%%%%%%%%%%%%%%%%%%%%%%%%%

%%%%%%%%%%%%%%%%%%%%%%%%%%%%%%%%%%%%%%%%%%%%%%%
\PART{Crystal}
%%%%%%%%%%%%%%%%%%%%%%%%%%%%%%%%%%%%%%%%%%%%%%%
\problem{3+1}{1/2 page}
\subproblem Give a recursive definition of the set S of bit strings with no more than a single 1 in them such as 00010, 010,000...
\subproblem Is $0010\in S$? Why?

\begin{solution}
\subsolution
\begin{itemize}
\item Base cases: the empty string $\epsilon$, 0, 1 $\in S$
\item Constructor Rule: If w is in S, then
\subitem C1: $0w\in S$
\subitem C2: $w0\in S$
\item “Nothing else” (generally implicit): Nothing is in S unless it is obtained from the base case and constructor rule.
\end{itemize}
\subsolution Yes, because
\begin{itemize}
\item $1\in S$ (Base case)
\item $10\in S$ by C2 
\item $010\in S$ by C1
\item $0010\in S$ by C1
\end{itemize}
\end{solution}

\problem{2+2}{1/2 page}
Let A= $\left\{ {5n|n\in\mathbb{N}}\right\}$ and S be the set defined as follows:
\begin{itemize}
\item Base Case: $5 \in S$
\item Constructor Rule: If $x\in S$ and $y\in S$, $(x+y) \in S$ 
\end{itemize}
\subproblem Use induction to prove that $A\subset S $.
\subproblem Use structural induction to prove that $S\subset A$.

\begin{solution}
\subsolution Proving that $A\subset S$ is equivalent to proving that $\forall x[x\in A\rightarrow x\in S]$
\begin{itemize}
\item Let P(n): $5n\in S$. We must show that for all $n\in\mathbb{N}, P(n)$.
\item Base case: When n=1, $5(1)=5\in S$ by base case in the definition of S.
\item Induction step: \\
Assuming for some $n\in\mathbb{N}$ P(n) is true, i.e. $5n\in S$, we want to prove that $P(n+1): 5(n+1)=5n+5\in S$.\\
Since 5, 5n$\in S$, $5n+5\in S$ by the constructor rule of definition of S.
\end{itemize}
\subsolution Proving that $S\subset A$ is equivalent to proving that $\forall x[x\in S\rightarrow x\in A]$
\begin{itemize}
\item Basis step: By the base case of the definition of S, $3\in S$. Since 3=3(1), $3\in A$
\item Recursive step: \\
Now consider the constructor rule in the definition of S. Assume elements $x,y\in S$ are also in A. We must  show that $x+y\in A$.\\
Since $x,y\in A$, x=3i and y=3j for some natural numbers i and j. So x+y=3i+3j=3(i+j), where $i+j\in\mathbb{N}$ since $i,j\in\mathbb{N}$. Thus $x+y\in A.$
\end{itemize}
\end{solution}

\end{document}
