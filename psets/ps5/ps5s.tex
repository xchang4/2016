%% Please fill in your name and collaboration statement here.
\newcommand{\studentName}{**FILL IN YOUR NAME HERE**}
\newcommand{\collaborationStatement}{**FILL IN YOUR COLLABORATION STATEMENT HERE \\ (See the syllabus for information)**}


%%%%%%%%%%%%%%%%%%%%%%%%%%%%%%%%%%%%%%%%%%%%%%%
\documentclass[solution, letterpaper]{cs20}
\usepackage{enumerate}
\usepackage{tikz}
\usepackage{pgf}
\usepackage{tikz}
\usepackage{hyperref}
\begin{document}
\header{5}{Due Wednesday, March 9, 2016 at 9:59am. All students should submit an electronic copy.}

%%%%%%%%%%%%%%%%%%%%%%%%%%%%%%%%%%%%%%%%%%%%%%%
\PART{Ben}
%%%%%%%%%%%%%%%%%%%%%%%%%%%%%%%%%%%%%%%%%%%%%%%
\problem{4}{1/3 page}
Show, by giving an example for each case, that the intersection of two uncountable sets can be: empty, finite, countably
infinite, or uncountably infinite.

\begin{solution}
An example of the empty set resulting from the intersection of two uncountable sets is \fbox{$\big[0,1\big] \cap \big[2,3\big] = \emptyset$}.
\newline

\noindent An example of a finite set resulting from the intersection of two uncountable sets is 

\noindent \fbox{$\big[0,1\big] \cap \big[1,2\big] = \{1\}$}.
\newline

\noindent An example of a countably infinite set resulting from the intersection of two uncountable sets is \fbox{$(\mathbb{Z} \cup \big[0,1\big]) \cap (\mathbb{Z} \cup \big[2,3\big]) = \mathbb{Z}$}.
\newline

\noindent An example of an uncountable set resulting from the intersection of two uncountable sets is \fbox{$\mathbb{R} \cap \mathbb{R}=\mathbb{R}$}.
\end{solution}

\problem{4+2}{1 page}
\subproblem The Schr\"{o}der-Bernstein Theorem states that for sets $S$ and $T$, if there exist injective functions $f : S \rightarrow T$ and $g : T \rightarrow S$, then $S$ and $T$ have the same cardinality. Using the Schr\"{o}der-Bernstein Theorem, show that the cardinality of the set of all real numbers between $0$ and $100$, inclusive, is the same as the cardinality of the set of all real numbers between, but not including, $0$ and $100$.

\subproblem Prove the finite case for the Schr\"{o}der-Bernstein Theorem. That is, prove that for finite sets $S$ and $T$, if there exist injective functions $f : S \rightarrow T$ and $g : T \rightarrow S$, then sets $S$ and $T$ have the same cardinality. (HINT: Check out the warm-up proof from the Relations and Functions lesson for some inspiration!)

\begin{solution}
\subsolution We want to show that the cardinality of $\big[0,100\big]$ is the same as $(0,100)$. Using the Schr\"{o}der-Bernstein Theorem, we want to construct an injective function from each set to the other. We can set the injective function $f : (0, 100) \rightarrow \big[0, 100\big]$ to be $f(x) = x$ for all $x \in (0, 100)$. If $f(x_1) = f(x_2)$, then $x_1 = x_2$, so $f$ is injective.

Next, we want to construct an injective function $g : \big[0, 100\big] \rightarrow (0, 100)$. The idea is to find a ``copy'' of $\big[0, 100\big]$ in $(0, 100)$, then do some scaling and translation to map $\big[0, 100\big]$ onto the copy. We find that $g(x) = \frac{x}{2} + 25$ is an appropriate function for $0 \leq x \leq 100$. If $0 \leq x \leq 100$, then $0 \leq \frac{x}{2} \leq 50$, so $25 \leq \frac{x}{2} + 25 \leq 75$. This proves that $g$ is a function from $\big[0, 100\big]$ to $\big[25, 75\big]$, which is a subset of $(0,100)$.

Next, we want to show that $g$ is injective. It suffices to show that if $g(x_1) = g(x_2)$, then $x_1 = x_2$. For $g(x_1) = g(x_2)$, we have: 
$$\frac{x_1}{2} + 25 = \frac{x_2}{2} + 25$$
$$\frac{x_1}{2} = \frac{x_2}{2}$$
$$x_1 = x_2$$
Having found satisfactory injective functions $f$ and $g$, we know by the Schr\"{o}der-Bernstein Theorem that the cardinality of $\big[0,100\big]$ is the same as $(0,100)$.

\subsolution We use proof by contradiction. Suppose that sets $S$ and $T$ do not have the same cardinality. Without loss of generality, assume that $|S| < |T|$. Now consider the injective function $g$. By the definition of an injective function, for any $a$ and $b$ in $T$, $a \neq b \Rightarrow g(a) \neq g(b)$. However, since $g$ is mapping elements from $T$ to elements in $S$ and there are more elements in $T$ than in $S$, by the Pigeonhole Principle there must be at least two unique elements in $T$ that map to the same element in $S$. In formal terms, $\exists a,b \in T $ s.t. $a \neq b \wedge g(a) = g(b)$. Then $g$ must not be an injective function, which is a contradiction, so we conclude that sets $S$ and $T$ must have the same cardinality (a symmetrical argument applies if $|S| > |T|$ by examining the injective function $f$).
\end{solution}

\problem{2+2}{1/2 page}
A robot wanders around a two-dimensional grid. The robot starts at (0,0) and is allowed to take four different types of steps:
\begin{enumerate}
\item (-2, +2)
\item (-4, +4)
\item (+1, -1)
\item (+3, -3)
\end{enumerate}
For example, the robot might take the following stroll. The types of steps are denoted by each arrow's subscript:
$$(0,0) \to_1 (-2,2) \to_3 (-1,1) \to_2 (-5,5) \to_4 (-2,2) \to \ldots$$

\subproblem Describe a state machine model of this problem.
\subproblem Will the robot ever reach (1, 2)? Either find an appropriate path for the robot or use the Invariant Principle to prove that no such path exists.

\begin{solution}
\subsolution In our state machine model, every state takes the form $(x,y)$. The start state is $(0,0)$, and the possible transitions out of a state $(x,y)$ are to $(x-2, y+2),(x-4,y+4),(x+1,y-1),(x+3,y-3)$.
\subsolution The robot will not.
We propose the following invariant: for any state $(x,y)$ 
\[x = -y \]
We see that it is true for the start state $(0,0)$, and all of the transitions preserve this invariant, as they add or subtract the same amount to both the $x$ and $y$ coordinates. However, our desired final state $(1,2)$ does not satisfy this invariant, and so it must be impossible to reach.
\end{solution}

%%%%%%%%%%%%%%%%%%%%%%%%%%%%%%%%%%%%%%%%%%%%%%%
\PART{Crystal}
%%%%%%%%%%%%%%%%%%%%%%%%%%%%%%%%%%%%%%%%%%%%%%%
\problem{3+1}{1/2 page}
\subproblem Using string concatenation as the sole constructor, give a recursive definition of the set $S$ of bit strings with no more than a single 1 in them 

(e.g. 00010, 010, or 000)
\subproblem Is $0010\in S$? How can you derive it from your base case?

\begin{solution}
\subsolution
\begin{itemize}
\item Base cases: the empty string $\epsilon$, 0, 1 $\in S$
\item Constructor Rule: If w is in S, then
\subitem C1: $0w\in S$
\subitem C2: $w0\in S$
\item ?Nothing else? (generally implicit): Nothing is in $S$ unless it is obtained from the base case and constructor rule.
\end{itemize}
\subsolution Yes, because
\begin{itemize}
\item $1\in S$ (Base case)
\item $10\in S$ by C2 
\item $010\in S$ by C1
\item $0010\in S$ by C1
\end{itemize}
\end{solution}

\problem{2+2}{1/2 page}
Let $A = \left\{ {5n:n\in\mathbb{N}}\right\}$ and let $S$ be the set defined as follows:
\begin{itemize}
\item Base Case: $5 \in S$
\item Constructor Rule: If $x\in S$ and $y\in S$, $x+y \in S$ 
\end{itemize}
\subproblem Use induction to prove that $A\subseteq S $.
\subproblem Use structural induction to prove that $S\subseteq A$.

\begin{solution}
\subsolution Proving that $A\subseteq S$ is equivalent to proving that $\forall x[x\in A\rightarrow x\in S]$
\begin{itemize}
\item Let P(n): $5n\in S$. We must show that for all $n\in\mathbb{N}, P(n)$.
\item Base case: When n=1, $5(1)=5\in S$ by base case in the definition of S.
\item Induction step: \\
Assuming for some $n\in\mathbb{N}$ P(n) is true, i.e. $5n\in S$, we want to prove that $P(n+1): 5(n+1)=5n+5\in S$.\\
Since 5, 5n$\in S$, $5n+5\in S$ by the constructor rule of definition of S.
\end{itemize}
\subsolution Proving that $S\subseteq A$ is equivalent to proving that $\forall x[x\in S\rightarrow x\in A]$
\begin{itemize}
\item Basis step: By the base case of the definition of S, $5\in S$. Since $5=5(1)$, $5\in A$
\item Recursive step: \\
Now consider the constructor rule in the definition of $S$. Assume elements $x,y\in S$ are also in $A$. We must show that $x+y\in A$.\\
Since $x,y\in A$, $x=5i$ and $y=5j$ for some natural numbers $i$ and $j$. So $x+y=5i+5j=5(i+j)$, where $i+j\in\mathbb{N}$ since $i,j\in\mathbb{N}$. Thus $x+y\in A.$
\end{itemize}
\end{solution}

\end{document}
