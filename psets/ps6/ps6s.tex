%% Please fill in your name and collaboration statement here.
\newcommand{\studentName}{**FILL IN YOUR NAME HERE**}
\newcommand{\collaborationStatement}{**FILL IN YOUR COLLABORATION STATEMENT HERE \\ (See the syllabus for information)**}


%%%%%%%%%%%%%%%%%%%%%%%%%%%%%%%%%%%%%%%%%%%%%%%
\documentclass[solution, letterpaper]{cs20}
\usepackage{enumerate}
\usepackage{tikz}
\usepackage{pgf}
\usepackage{tikz}
\usepackage{hyperref}
\begin{document}
\header{6}{Due Wednesday, March 23, 2016 at 9:59am. All students should submit an electronic copy.}

%%%%%%%%%%%%%%%%%%%%%%%%%%%%%%%%%%%%%%%%%%%%%%%
\PART{Michelle}

\problem{2+4}{8 lines}

\subproblem Determine the maximum number of edges in a DAG with $n$ vertices.
\subproblem Prove your answer using induction.  

\begin{solution}
\subsolution The maximum number of edges is is $n(n-1)/2$. 
\subsolution Base case: $n = 1$. Maximum number of edges is 0. 
Induction step: assume that for some DAG with $n$ vertices, the maximum number of edges is $n(n-1)/2$. We must prove that for $n+1$ vertices, the maximum number of edges will be $(n+1)(n)/2$. Let us consider our graph with $n$ vertices and add one additional vertex.  We want to maximize the number of edges in this new graph without creating a cycle. If we add an edge from our new vertex to every existing vertex in the graph, we will not create a cycle since there will be no way to return to our vertex. Thus, we will end up with $n(n-1)/2 + n$ edges, which is by simple algebra equal to $(n+1)(n)/2$. 
\end{solution}


\problem{5}{10 lines}
A directed graph $G$ has a Euler Tour if and only if there is a closed walk in that graph that uses each edge exactly once, although it may visit vertices multiple times. Prove that a graph has a Euler Tour iff the in and out degrees of every vertex are equal. 

\begin{solution}
We complete the proof in both directions. First, let us prove that if a graph $G$ has a Euler tour, then the in an out degrees of each vertex are equal. There are two cases: where our path passes through every vertex once and where it passes through some of these vertices multiple times. If it passes through every vertex once (a simple tour), then the in degree and out degree of each vertex is one; the in and out degree of every vertex is equal. We can express other other case as several Euler tours of subsets of these vertices combined; thus the in and out degrees will still be equal since each of these simple Euler tours will be equal. 

Now let us prove that if we have a graph with equal in and out degree, then we have a Euler tour. Proof by construction. Begin construction at any vertex in the graph that has out degree of at least 1 and begin constructing a path. For all reachable vertexes from V, since in and out degrees are equal, we know we will be able to leave a vertex if we reach it. Eventually we must end back up at the vertex we started at. If not every edge has been traversed, we repeat this construction starting at another vertex with at least one edge not yet traversed. We continue until every edge has been traversed; the Euler tour of the graph is the union of these smaller Euler tours produced at each step of the construction. 

\end{solution}
%%%%%%%%%%%%%%%%%%%%%%%%%%%%%%%%%%%%%%%%%%%%%%%

%%%%%%%%%%%%%%%%%%%%%%%%%%%%%%%%%%%%%%%%%%%%%%%
\PART{Jack}
%%%%%%%%%%%%%%%%%%%%%%%%%%%%%%%%%%%%%%%%%%%%%%%

\problem{1+1+1+1+1}{one line each}

Determine whether each of the possible binary relations is an equivalence relation. If you do not believe that the relation is an equivalence, you should justify your answer by stating the properties of equivalance relations that the relation fails to satisfy. Otherwise, no justification is needed.

\subproblem The relation $R$ where $aRb \iff$ a is a sibling of b.
\subproblem The relation $R$ where $aRb \iff$ a is friends with b.
\subproblem The relation $R$ where $aRb$ for $a, b \in \{$ Rock, Paper, Scissors $\} \iff a$ beats $b$.
\subproblem The relation $R = \{(a, a), (a, c), (b, b), (c, c), (c, a) \}$ on the set $A = \{a, b, c\}$.
\subproblem The relation $R = \{(a, b), (a, a), (c, c), (a, c), (c, a), (b, a), (b, b) \}$ on the set $A = \{a, b, c\}$.

\solution

\subsolution This is not an equivalence relation, since the reflexive property does not hold (you are not your own sibiling).
\subsolution This is not an equivalence relation, since the reflexive and transitive properties do not hold (you are not friends with yourself, two people who share a friend do not necessarily have to be friends).
\subsolution This is not an equivalance relation, since none of the properties hold.
\subsolution This is an equivalence relation.
\subsolution This is not an equivalence relation, since the transitive property does not hold ($bTa$ and $aTc$ but it is not the case that $bTc$.

\problem{1+2+2+2}{two lines each}

The following question refers to the set $S = \{T, F\}$.

\subproblem How many binary relations are there on $S$?
\subproblem List all of the reflexive binary relations on $S$
\subproblem List all of the asymmetric binary relations on $S$
\subproblem List all of the reflexive and symmetric binary relations on $S$

\solution

\subsolution 16
\subsolution $\{(T, T), (F, F)\}$, $\{(T, T), (F, F), (T, F)\}$, $\{(T, T), (F, F), (F, T)\}$, and $\{(T, T), (F, F), (T, F), (F, T)\}$
\subsolution $\emptyset$, $\{(T, F)\}$, and $\{(F, T)\}$
\subsolution $\{(T, T), (F, F)\}$ and $\{(T, T), (F, F), (T, F), (F, T)\}$

\problem{4}{six lines}

Consider the set $\mathbb{N} \times \mathbb{N}$ of all ordered pairs of positive integers. Show that the relation $(a, b) \sim (c,d) \iff ad = bc$ is an equivalence relation.

\solution

\subsolution Reflexifity: for a pair $(a, b)$ it is true that $ab = ba$, so $(a, b) \sim (a, b)$.
\subsolution Symmetry: if $(a, b) \sim (c, d)$, then $ad = bc$. This implies that $cb = da$ is true, so $(c, d) \sim (a, b)$.
\subsolution Transitivity: if $(a, b) \sim (c, d)$ and if $(c, d) \sim (e, f)$, then $ad = bc$ and $cf = de$. Multiplying the first equation by $f$ we get $adf = bcf$, and substituting for $cf$ from the second equation we get $adf = bde$. Since $d$ is not 0, we can cancel it from both sides to get $af = be$, which shows that $(a, b) \sim (e, f)$.

\end{document}
