%% Please fill in your name and collaboration statement here.
\newcommand{\studentName}{**FILL IN YOUR NAME HERE**}
\newcommand{\collaborationStatement}{**FILL IN YOUR COLLABORATION STATEMENT HERE \\ (See the syllabus for information)**}


%%%%%%%%%%%%%%%%%%%%%%%%%%%%%%%%%%%%%%%%%%%%%%%
\documentclass[solution, letterpaper]{cs20}
\usepackage{enumerate}
\usepackage{tikz}
\usepackage{pgf}
\usepackage{tikz}
\usepackage{hyperref}
\begin{document}
\header{6}{Due Wednesday, March 23, 2016 at 9:59am. All students should submit an electronic copy.}

%%%%%%%%%%%%%%%%%%%%%%%%%%%%%%%%%%%%%%%%%%%%%%%
\PART{Michelle}

\problem{2+4}{8 lines}

\subproblem Determine the maximum number of edges in a DAG with $n$ vertices.
\subproblem Prove your answer using induction.  

\begin{solution}
\subsolution The maximum number of edges is is $n(n-1)/2$. 
\subsolution Base case: $n = 1$. Maximum number of edges is 0. 
Induction step: assume that for some DAG with $n$ vertices, the maximum number of edges is $n(n-1)/2$. We must prove that for $n+1$ vertices, the maximum number of edges will be $(n+1)(n)/2$. Let us consider our graph with $n$ vertices and add one additional vertex.  We want to maximize the number of edges in this new graph without creating a cycle. If we add an edge from our new vertex to every existing vertex in the graph, we will not create a cycle since there will be no way to return to our vertex. Thus, we will end up with $n(n-1)/2 + n$ edges, which is by simple algebra equal to $(n+1)(n)/2$. 
\end{solution}


\problem{5}{10 lines}
An Euler Tour on a directed graph is a closed walk that uses each edge exactly once. Note that an Euler Tour may visit vertices multiple times. Prove that if a directed graph $G = (V, E)$ has an Euler tour, then for all $v \in V$, the in degree of $v$ is equal to the out degree of $v$. (Hint: How does removing all of the edges in a cycle affect the differences between in and out degrees?)

\begin{solution}
Let $G = (V, E)$ be a directed graph with an Euler tour. Let $v_1, ..., v_m \in V$ be the vertices visited in the Euler tour. Note that $v_1 = v_m$ by the definition of a closed walk. Starting at $v_1$, follow the walk until you find a vertex $v_j$ that has already been seen, i.e. there is a $v_i$ with $i < j$ and $v_i = v_j$. Consider the edges in the original walk that visit $v_i, v_{i+1}, ..., v_j$. Since $v_j$ is the first non-unique vertex, these edges must form a cycle. Using the hint, we know that removing these edges will preserve the differences between the in and out degrees of each of the vertices in the cycle. Remove these edges. Continue along the original Euler tour until another repeated vertex is encountered, and repeat the same edge removal procedure. This process will conclude at $v_1$ when there are no edges in the graph remaining. At this point, all in and out degrees are 0, and since we never modified the differences between in and out degrees, the original in and out degrees must have been equivalent for all vertices.

\end{solution}
%%%%%%%%%%%%%%%%%%%%%%%%%%%%%%%%%%%%%%%%%%%%%%%

%%%%%%%%%%%%%%%%%%%%%%%%%%%%%%%%%%%%%%%%%%%%%%%
\PART{Jack}
%%%%%%%%%%%%%%%%%%%%%%%%%%%%%%%%%%%%%%%%%%%%%%%

\problem{1+1+1+1+1}{one line each}

Determine whether each of the possible binary relations is an equivalence relation. If you do not believe that the relation is an equivalence, you should justify your answer by stating the properties of equivalence relations that the relation fails to satisfy. Otherwise, no justification is needed.

\subproblem The relation $R$ where $aRb \iff$ a is a sibling of b.
\subproblem The relation $R$ where $aRb \iff$ a is friends with b.
\subproblem The relation $R$ where $aRb$ for $a, b \in \{$ Rock, Paper, Scissors $\} \iff a$ beats $b$.
\subproblem The relation $R = \{(a, a), (a, c), (b, b), (c, c), (c, a) \}$ on the set $A = \{a, b, c\}$.
\subproblem The relation $R = \{(a, b), (a, a), (c, c), (a, c), (c, a), (b, a), (b, b) \}$ on the set $A = \{a, b, c\}$.

\begin{solution}

\subsolution This is not an equivalence relation, since the reflexive property does not hold (you are not your own sibling).
\subsolution This is not an equivalence relation, since the reflexive and transitive properties do not hold (you are not friends with yourself, two people who share a friend do not necessarily have to be friends).
\subsolution This is not an equivalence relation, since none of the properties hold.
\subsolution This is an equivalence relation.
\subsolution This is not an equivalence relation, since the transitive property does not hold ($bTa$ and $aTc$ but it is not the case that $bTc$.

\end{solution}

\problem{4}{six lines}

Consider the set $\mathbb{N} \times \mathbb{N}$ of all ordered pairs of positive integers. Show that the relation $(a, b) \sim (c,d) \iff ad = bc$ is an equivalence relation.

\begin{solution}

\subsolution Reflexivity: for a pair $(a, b)$ it is true that $ab = ba$, so $(a, b) \sim (a, b)$.
\subsolution Symmetry: if $(a, b) \sim (c, d)$, then $ad = bc$. This implies that $cb = da$ is true, so $(c, d) \sim (a, b)$.
\subsolution Transitivity: if $(a, b) \sim (c, d)$ and if $(c, d) \sim (e, f)$, then $ad = bc$ and $cf = de$. Multiplying the first equation by $f$ we get $adf = bcf$, and substituting for $cf$ from the second equation we get $adf = bde$. Since $d$ is not 0, we can cancel it from both sides to get $af = be$, which shows that $(a, b) \sim (e, f)$.
\end{solution}

\end{document}
