%% Please fill in your name and collaboration statement here.
\newcommand{\studentName}{**FILL IN YOUR NAME HERE**}
\newcommand{\collaborationStatement}{**FILL IN YOUR COLLABORATION STATEMENT HERE \\ (See the syllabus for information)**}


%%%%%%%%%%%%%%%%%%%%%%%%%%%%%%%%%%%%%%%%%%%%%%%
\documentclass[solution, letterpaper]{cs20}
\usepackage{enumerate}
\usepackage{tikz}
\usepackage{pgf}
\usepackage{tikz}
\usepackage{hyperref}
\begin{document}
\header{6}{Due Wednesday, March 23, 2016 at 9:59am. All students should submit an electronic copy.}

%%%%%%%%%%%%%%%%%%%%%%%%%%%%%%%%%%%%%%%%%%%%%%%
\PART{Michelle}
%%%%%%%%%%%%%%%%%%%%%%%%%%%%%%%%%%%%%%%%%%%%%%%

%%%%%%%%%%%%%%%%%%%%%%%%%%%%%%%%%%%%%%%%%%%%%%%
\PART{Jack}
%%%%%%%%%%%%%%%%%%%%%%%%%%%%%%%%%%%%%%%%%%%%%%%

\problem{1+1+1+1+1}{one line each}

Determine whether each of the possible binary relations is an equivalence relation. If you do not believe that the relation is an equivalence, you should justify your answer by stating the properties of equivalance relations that the relation fails to satisfy. Otherwise, no justification is needed.

\subproblem The relation $R$ where $aRb \iff$ a is a sibling of b.
\subproblem The relation $R$ where $aRb \iff$ a is friends with b.
\subproblem The relation $R$ where $aRb$ for $a, b \in \{$ Rock, Paper, Scissors $\} \iff a$ beats $b$.
\subproblem The relation $R = \{(a, a), (a, c), (b, b), (c, c), (c, a) \}$ on the set $A = \{a, b, c\}$.
\subproblem The relation $R = \{(a, b), (a, a), (c, c), (a, c), (c, a), (b, a), (b, b) \}$ on the set $A = \{a, b, c\}$.

\solution

\subsolution This is not an equivalence relation, since the reflexive property does not hold (you are not your own sibiling).
\subsolution This is not an equivalence relation, since the reflexive and transitive properties do not hold (you are not friends with yourself, two people who share a friend do not necessarily have to be friends).
\subsolution This is not an equivalance relation, since none of the properties hold.
\subsolution This is an equivalence relation.
\subsolution This is not an equivalence relation, since the transitive property does not hold ($bTa$ and $aTc$ but it is not the case that $bTc$.

\problem{3+3+3}{three lines each}

Recall that a binary relation $R \subseteq A \times A$ is the set of pairs of elements in $A$ that satisfy the relation. Let $R_1$ and $R_2$ be arbitrary equivalence relations on a set $A$. For each of the following, determine whether it must also be an equivalence relation. If so, prove it. If not, give a counterexample.

\subproblem $R_1 \cup R_2$
\subproblem $R_1 \cap R_2$
\subproblem $R_1 \setminus R_2$

\solution

\subsolution
\subsolution
\subsolution

\problem{4}{three lines each}

Consider the set $\mathbb{N} \times \mathbb{N}$ of all ordered pairs of positive integers. Show that the relation $(a, b) \sim (c,d) \iff ad = bc$ is an equivalence relation.

\solution

\subsolution Reflexifity: for a pair $(a, b)$ it is true that $ab = ba$, so $(a, b) \sim (a, b)$.
\subsolution Symmetry: if $(a, b) \sim (c, d)$, then $ad = bc$. This implies that $cb = da$ is true, so $(c, d) \sim (a, b)$.
\subsolution Transitivity: if $(a, b) \sim (c, d)$ and if $(c, d) \sim (e, f)$, then $ad = bc$ and $cf = de$. Multiplying the first equation by $f$ we get $adf = bcf$, and substituting for $cf$ from the second equation we get $adf = bde$. Since $d$ is not 0, we can cancel it from both sides to get $af = be$, which shows that $(a, b) \sim (e, f)$.

\end{document}
