%% Please fill in your name and collaboration statement here.
\newcommand{\studentName}{**FILL IN YOUR NAME HERE**}
\newcommand{\collaborationStatement}{**FILL IN YOUR COLLABORATION STATEMENT HERE \\ (See the syllabus for information)**}


%%%%%%%%%%%%%%%%%%%%%%%%%%%%%%%%%%%%%%%%%%%%%%%
\documentclass[solution, letterpaper]{cs20}
\usepackage{enumerate}
\usepackage{tikz}
\usepackage{pgf}
\usepackage{tikz}
\usepackage{hyperref}
\begin{document}
\header{2}{Due Wednesday, February 10, 2016 at 9:59am. All students should submit an electronic copy.}


%%%%%%%%%%%%%%%%%%%%%%%%%%%%%%%%%%%%%%%%%%%%%%%
\PART{Jack}
%%%%%%%%%%%%%%%%%%%%%%%%%%%%%%%%%%%%%%%%%%%%%%%

\problem{3+3+4+4}{3 lines each}

\subproblem Put the following formula in conjunctive form: $(p \leftrightarrow q) \land r$
\subproblem Put the following formula in disjunctive form: $(p \oplus q) \rightarrow (r \land q)$
\subproblem Put the following formula in disjunctive form: $((p \rightarrow q) \rightarrow p) \rightarrow p$. What do you notice about your answer?
\subproblem Prove that any propositional formula can be written in conjunctive normal form. 

\begin{solution}
\subsolution  We can replace $p \leftrightarrow q$ with $(p \rightarrow q) \land (q \rightarrow p)$. The statement then becomes $(p \rightarrow q) \land (q \rightarrow p) \land r$, which we can rewrite as $(\lnot p \lor q) \land (\lnot q \lor p) \land r$, which is in CNF.

\subsolution Recall that $p \oplus q \equiv (p \lor q) \land \lnot (p \land q)$. The expression then expands to $(p \lor q) \land \lnot (p \land q) \rightarrow (r \land q)$ which is equivalent to $\lnot((p \lor q) \land \lnot (p \land q)) \lor (r \land q)$. Using De Morgan's laws to remove the negation, we get $\lnot (p \lor q) \lor (p \land q) \lor (r \land q)$, which simplifies to $(\lnot p \land \lnot q) \lor (p \land q) \lor (r \land q)$.

\subsolution Eliminating $\rightarrow$, we get $\lnot (\lnot (\lnot p \lor q) \lor p) \lor p$. Then, pushing the negations inwards, the expression becomes $(\lnot (\lnot p \lor q) \land \lnot p) \lor p$ and $((\lnot p \lor q) \land \lnot p) \lor p$. Finally, pushing the disjunctions in using the distributive property gives $(\lnot p \lor q \lor p) \land (\lnot p \lor p)$. This is a tautology, known as Peirce's law.

\subsolution Any logical statement can be represented using a truth table, where each possible combination of variables evaluates to either true or false. We can say that for any combination of variables the statement is true as long as it is not one of the combinations that evaluates to false. The previous sentence can always be represented as follows: reading off the truth table, each term of the  statement will be the negation of one of the cases where the statement evaluates to false. Each of these terms will be joined with ands. Logically this statement will be correct, but it is not yet in conjunctive normal form because there are ands inside the individual terms, as well as connecting the terms. Conjunctive normal form needs to have exclusively ors inside each of the terms. 

De Morgan's law can then be applied to each of the individual terms in order to convert the statement to conjunctive normal form. When De Morgan's law is applied to one term, all of the variables in that term will flip their value (true to false, or false to true), the term as a whole will be negated, and the ands joining the variables will flip to ors. If De Morgan's law is applied to each of the terms in the statement as just described, the resultant statement will have terms joined with ands but there will only be ors within each individual term; the statement will be in conjunctive normal form. Since any logical statement can be represented using a truth table, this methodology can be used for any logical statement. 

\end{solution}

%%%%%%%%%%%%%%%%%%%%%%%%%%%%%%%%%%%%%%%%%%%%%%%
\PART{Michelle}
%%%%%%%%%%%%%%%%%%%%%%%%%%%%%%%%%%%%%%%%%%%%%%%

\problem{1+2}{}

While wandering the halls of Hogwarts, Harry stumbles upon a note that reads: "Danger lies before you, while safety lies behind. Drink some of us to help you, the combination you must find". Harry sees four bottles of mysterious potion, one colored red, one blue, one yellow, and one green. A sign next to the door reads:

\begin{enumerate}
\item Drink two of these potions, one must be colored blue.
\item If blue you do not have, the other three will tide you through.
\item Pair not blue with red
\item Or face your fate with dread
\end{enumerate}

Luckily, Harry sees all the colored bottles. Define r, y, b, g:
\begin{enumerate}
\item r: Grace drinks the red potion
\item y: Grace drinks the yellow potion
\item b: Grace drinks the blue potion
\item g: Grace drinks the green potion
\end{enumerate}

\subproblem Write a proposition that evaluates to True if Harry selects the a valid combination of potions. 
\subproblem Draw a logic circuit that implements the proposition in (a) with as few gates as possible from this list: \textsc{Not}, \textsc{Or}, \textsc{And}, \textsc{Xor}, \textsc{Nand}, \textsc{Xor}, \textsc{Xnor}.

\begin{solution}

\subsolution $(r \land y \land g) \lor (b \land (y \oplus g))$

could also do 
$(r \land y \land g) \lor (b \land y) \lor (b \land g)$
\subsolution [several possible solutions for this]

\end{solution}

\problem

The Boolean function $G(p,q,r)$ is defined in the table below. You can think of the table as a truth table where the last column is the value of some unknown compound proposition consisting of the variables p, q, and r. Additionally, ``0'' represents False and ``1'' represents True. 


\begin{table}[h]
\centering

\begin{tabular}{| c c c | c |}
\hline
p & q & r & G(p,q,r) \\ \hline
0 & 0 & 0 & 0 \\ 
0 & 0 & 1 & 0 \\ 
0 & 1 & 0 & 0 \\ 
0 & 1 & 1 & 1 \\ 
1 & 0 & 0 & 0 \\ 
1 & 0 & 1 & 1 \\ 
1 & 1 & 0 & 0 \\ 
1 & 1 & 1 & 0 \\ \hline

\end{tabular}
\end{table}

\subproblem Construct a proposition in disjunctive normal form whose value is the last column of the truth table. 
\subproblem Show that the proposition in (a) is equivalent to $(p \oplus q)\land r$.
\subproblem How many logic gates would be required to construct a circuit for the expression in (a), assuming you didn't simplify it? How many for the simplified expression in (b)?  In both cases, you may use any types of logic gates you wish.

\begin{solution}

\subsolution
Writing one term for each row in which $G(p, q, r)$ is true, we conclude:
$$G(p, q, r) \equiv (\lnot p \land q \land r) \lor (p \land \lnot q \land r)$$

\subsolution
Using the identity $p \oplus q \equiv (p \land \lnot q) \lor (\lnot p \land q)$, we can write:
  \begin{align*}
      (p \oplus q) \land r & \equiv ((p \land \lnot q) \lor (\lnot p \land q)) \land r \\
                           & \equiv (p \land \lnot q \land r) \lor (\lnot p \land q \land r) \\
                           & \equiv (\lnot p \land q \land r) \lor (p \land \lnot q \land r)
  \end{align*}
\emph{Note:  This problem can also be solved by showing that the two expressions have the same truth table.}

\subsolution
For the expression in (a), you would need 1 \textsc{Or}, 2 \textsc{Not} and 4 \textsc{And} gates, for a total of 7 logic gates.
For the expression in (b), you would need 2 logic gates, 1 \textsc{And} and 1 \textsc{Xor}.

\end{solution}

\end{document}
