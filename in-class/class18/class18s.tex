\documentclass[solution, letterpaper]{cs20inclass}
\usepackage{enumerate}
\usepackage{tikz}
\usepackage{pgf}
\usepackage{tikz}
\usepackage{hyperref}
\begin{document}
\header{6}{Wednesday, March 9, 2016}

\noindent Author: Michelle Danoff, Tom Silver% \\

\paragraph*{Executive Summary}
\begin{enumerate}
\item Properties of binary relations
\begin{itemize}
\item \textit{Transitive}: A binary relation $R$ on the set $A$ is transitive iff\\$u R v \wedge v R w \implies u R w$ for all $u,v,w \in A$.
\item \textit{Reflexive}: $u R u$ for all $u \in A$.
\item \textit{Irreflexive}: $\neg(uRu)$ for all $u \in A$
\item \textit{Symmetric}: $u R w \implies w R u$ for all $u,w \in A$.
\item \textit{Antisymmetric}: $u R w \implies \neg(wRu)$ for all $u, w \in A$, $u \neq w$.
\item \textit{Asymmetric}: $u R w \implies \neg(wRu)$ for all $u, w \in A$.
\end{itemize}
\item Recall that $G$ is a binary relation on $V$, where $uGw$ means that there is an edge from $u$ to $w$.
\begin{itemize}
\item $G^+$ is transitive and is the \textit{transitive closure} of $G$. This means that $G^+$ is the minimal transitive relation that includes $G$ (i.e. $G \subseteq G^+$).
\item $G^*$ is reflexive, transitive, and the \textit{reflexive transitive closure} of $G$.
\end{itemize}
\item The vertices $u,v \in V$ are \textit{strongly connected} iff $uG^*v \wedge vG^*u$. That is, if there exists a walk from $u$ to $v$ and a walk back from $v$ to $u$.
\item Special types of relations
\begin{itemize}
\item \textit{Strict partial orders}: transitive and irreflexive
\item \textit{Weak partial orders}: transitive, reflexive, and antisymmetric
\item \textit{Equivalence relations}: transitive, reflexive, and symmetric
\item A relation $R$ is a weak partial order iff $R = D^*$ for some DAG $D$
\item A relation $R$ is a equivalence relation iff $R$ is the strongly connected relation of some digraph
\end{itemize}
\item An equivalence relation $R$ decomposes the domain into subsets called \textit{equivalence classes}, where $aRb$ iff $a$ and $b$ are in the same equivalence class.
\end{enumerate}

\problem
Draw one directed graph with 3 vertices $A, B, C$ for each of the following relationships
\subproblem Reflexive
\subproblem Symmetric
\subproblem Antisymmetric
\subproblem Transitive


\begin{solution}
\includegraphics[width=3in]{3NodeGraphs.jpg}



\end{solution}

\problem
Prove that if a relation $R$ is transitive and irreflexive, then it is asymmetric.

\begin{solution}
Proof by contradiction. Assume for a moment the graph is symmetric. Let us consider two connected nodes in the graph $a$ a $b$. If there is an edge from $a$ to $b$  then there must also be edges from $b$ to $a$ since the graph is symmetric. Since the graph is transitive, there also be an edge from $a$ to $a$. However, we now have a contradiction since the graph is irreflexive. The graph must be asymmetric. 


\end{solution}

\problem Say that a string $x$ overlaps a string $y$ if there exist strings $p,q,r$ such that $x = pq$ and $y = qr$, with $q \neq \epsilon$. For example, $abcde$ overlaps $cdefg$, but does not overlap $bcd$ or $cdab$. Answer each of the following questions and prove your answer, or provide a counterexample.

\subproblem Is the overlap relation reflexive? 
\subproblem Is it symmetric?
\subproblem Is it transitive?

\begin{solution}
\subsolution Yes. A string will always overlap with itself, the entire string becomes the q section. 
\subsolution No. Counterexample: consider strings $abc$ and $bcx$. These strings overlap, but $bcx$ and $abc$ do not overlap. 
\subsolution No. Consider strings $abc$, $cde$, $efg$, $abc$ and $cde$ overlap, as do $cde$ and $efg$. However, $abc$ and $efg$ do not overlap. 

 
\end{solution}

\problem Determine what properties each of the following relations have. For those that are equivalence relations, briefly describe what the equivalence classes are in the relation.

\subproblem The relation ``shares a class with'', where two people share a class if there is a class they are both enrolled in this semester.
\subproblem The relation $R$ on $\mathbb{Z}$, where $aRb$ if $b$ is a multiple of $a$.
\subproblem The relation $R$ on $\mathbb{Z} \times \mathbb{Z}$, with $(a,b)$ $R$ $(c,d)$ if $ad = bc$. 


\begin{solution}
\subsolution Reflexive: you always share a class with yourself. Symmetric: if you are taking the same class as another person, then they are taking a class with you. NOT transitive. 
\subsolution Reflexive: a number always is a multiple of itself.  Transitive: consider $aRb$ and  $bRc$ $b = ax$, $c = by$. Thus, $c = axy$, where $xy$ is some multiplier. Not necessarily symmetric. 
\subsolution Reflexive: for $(a,b)$ $R$ $(a,b)$, $ab = ab$. Transitive: consider $(a,b)$ $R$ $(c,d)$ and  $(c,d)$ $R$ $(e,f)$. Then $ad = bc$ and $cf = de$. Then $c/d$ = $a/b$ = $e/f$. Symmetric: $cb = ad$, order does not matter. This is an equivalence relationship between sets of tuples for which this property applies. 

\end{solution}




\end{document}
