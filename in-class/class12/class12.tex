\documentclass[solution, letterpaper]{cs20inclass}
\usepackage{enumerate}
\usepackage{tikz}
\usepackage{pgf}
\usepackage{tikz}
\usepackage{hyperref}
\begin{document}
\header{12}{Wednesday, February 24, 2016}

\noindent Author: Hannah Blumberg

\paragraph*{Executive Summary}

\begin{itemize}
  \item A set is simply a ``bunch of objects.'' The objects that comprise a set are called its {\it elements} or {\it members}.
  \item A set is determined by its elements: two sets are equal if and only if they have exactly the same elements. For this reason, sets are not inherently ordered and elements of a set cannot ``appear more than once" in the set.
  \item A set $A$ is a {\em subset} of a set $B$, denoted by $A \subseteq B$, if every object in A is also in B.
  \item The {\em union} of two sets $A$ and $B$, denoted by $A\cup B$, is the set of all objects that are either in $A$ or in $B$ (or both).
  \item The {\em intersection} of two sets $A$ and $B$, denoted by $A \cap B$, is the set of all objects that are in both $A$ and $B$.
  \item The {\em product} of two sets $A$ and $B$, denoted by $A\times B$, is the set of all ordered pairs $(x,y)$ such that $x$ is an element of $A$ (written $x \in A$) and $y$ is an element of $B$ (written $y \in B$).
  \item The {\em difference} of $A$ and $B$, denoted by $A-B$ or $A\setminus B$, is the set of elements that are in $A$ but are not in $B$.
  \item The {\em complement} of a set $X$ (in a domain $D$) is the set $\overline X = \{y \in D : y \notin X\}$ or $\overline X = \{y \in D : \neg (y \in X)\}$ containing all objects that are not elements of $X$.
  \item The {\em power set} of $A$, denoted by $P(A)$, is the set of all subsets of $A$.
  \item We often use {\em set-builder notation} as a concise way to describe what it means for a particular element $x$ to be a member of a set $S$, denoting the set of all $x\in D$ satisfying the property $P$ by $\{x\in D\mid P(x)\}$.
  \item A set is {\em finite} if it can be put in one-to-one correspondence with a bounded sequence of natural numbers ($1,\dots,n$). 
  \item The {\em cardinality} of a finite set $A$, written $|A|$, is the number of elements in that set.
\end{itemize}

\pagebreak

%% PROBLEM 1 %%
\problem Let $A = \{1,2,3,4,5\}$ and $B = \{a,b,c,d,4,5\}$.

\subproblem What is $A \cap B$?
\subproblem What is $A \cup B$?
\subproblem What is $|A \times B|$?
\subproblem What is $|P(A)|$?

\begin{solution}
  % Write your answer here.
\end{solution}


%% PROBLEM 2 %%
\problem
Using set-builder notation, give formal descriptions of the following sets:

\subproblem The product of two sets $X$ and $Y$.
\subproblem The difference between two sets $X$ and $Y$.
\subproblem The power set of a set $X$, denoted $P(X)$.

\begin{solution}
  % Write your answer here.
\end{solution}


%% PROBLEM 3 %%
\problem

\subproblem Explain why $|X \cup Y| \neq |X| + |Y|$. 
\subproblem Provide a formula for $|X \cup Y|$ in terms of $|X|$, $|Y|$, and $|X \cap Y|$.
\subproblem (BONUS) Generalize this formula to unions of more than two sets.

\begin{solution}
  % Write your answer here.
\end{solution}


%% PROBLEM 4 %%
\problem
(BONUS) There are 100 students enrolled in at least one of the following classes: CS20, CS51, and Math 21b. There are 60 students enrolled in CS20, 70 students enrolled in CS51, 30 students enrolled in Math 21b, and 10 students enrolled in all three classes.

\subproblem Let $A$, $B$, and $C$ represent the set of all students in CS20, CS51, and Math 21b respectively. Represent the information given above using set union, intersection, and cardinality.
\subproblem How many students are enrolled in exactly two of the classes?

\begin{solution}
  % Write your answer here.
\end{solution}

\end{document}
