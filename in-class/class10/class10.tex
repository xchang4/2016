\documentclass[solution, letterpaper]{cs20inclass}
\usepackage{enumerate}
\usepackage{tikz}
\usepackage{pgf}
\usepackage{tikz}
\usepackage{hyperref}
\begin{document}
\header{10}{Wednesday, February 17, 2016}

\noindent Author: Tom Silver% \\

\paragraph*{Executive Summary}
\begin{enumerate}

\item Propositions and predicates
\begin{itemize}

\item A proposition $P$ is like a boolean variable. Its value is either ``true'' or ``false.''
\item A predicate $P(x)$ is like a boolean-valued function. It may have the value ``true'' for some values of the $x$ and the value ``false'' for others.
\item A predicate can have more than one argument; e.g. Enrolled($x, y$) might mean ``student $x$ is enrolled in course $y$.'' 
\item In principle it is important to know the (possibly infinite) set of values $D$ that $x$ and $y$ might assume (the domain of the function). 
\end{itemize}

\item Quantifiers
\begin{itemize}
\item The existential quantifier: $\exists x P(x)$ or $\exists x. P(x)$ or $\exists x \mbox{ s.t. }P(x)$ means ``there exists at least one $x$ in the domain $D$ such that $P(x)$ is true.

\item The universal quantifier: $\forall x P(x)$ or $\forall x. P(x)$  means ``for every $x$ in the domain $D$, $P(x)$ is true.

\end{itemize}

\item Multiple quantifiers
\begin{itemize}

\item  $\exists x P(x)$ and $\forall x P(x)$ are both propositions, subject to the rules of logic that you already know.

\item  $\exists x P(x,y)$ and $\forall x P(x,y)$ are both predicates of the form $Q(y)$, subject to the rules of quantificational logic that you are learning.

\item $\exists x \exists y P(x,y)$ and $\forall x \forall y P(x,y)$ are both propositions. The order of the quanitifiers is irrelevant.

\item $\exists x \forall y P(x,y)$ and $\forall y \exists x P(x,y)$ are both propositions, but they are different! The order of the quantifiers is important. 

\end{itemize}

\item Negation and quantifiers
\begin{itemize}
\item $\neg (\exists x. P(x)) \leftrightarrow \forall x. (\neg P(x))$.
\item $\neg (\forall x. P(x)) \leftrightarrow \exists x. (\neg P(x))$.

\end{itemize}

\end{enumerate}

\problem Recall from Problem Set 1: Let $A$ be the set of your pigeons, and let $B$ be the set of pigeonholes in which they live. The \textit{Generalized Pigeonhole Principle} states that for a natural number $k$, if $|A| > k|B|$, then there is a pigeonhole in which more than $k$ pigeons live. Restate the GPP using quantifiers. Then negate the GPP (as you would for a proof by contradiction).

\begin{solution}
  % Write your answer here.
\end{solution}

\problem
 The domain of discourse is the set of all finite-length binary strings. The predicates Sub($x,y$) (meaning $x$ is a substring of $y$) and Pre($x,y$) (meaning $x$ is a prefix of $y$) are available.

\subproblem  Write an expression that means $x$ consists of alternating 0s and 1s, e.g 01010 or 101010.
\subproblem Write two different expressions that mean $x$ consists of one or more 1s, and no 0s. Important caveat: neither of your expressions may contain ``0''.
\subproblem (BONUS) Write two additional expression for (c) under the same constraint.

\begin{solution}
  % Write your answer here.
\end{solution}


\problem
(BONUS) We define a committee to be a subset of senators $S = \{s_1,s_2,\cdots,s_n\}$. The predicate $M(s,C)$ means ``Senator $s$ is a member of committee $C$.'' Rewrite the following in terms of predicate logic. You may use ``$=$'' and ``$\in$'' in your expressions.

\subproblem Every committee has at least two senators serving on it.
\subproblem No two senators serve on more than one committee together.

\begin{solution}
  % Write your answer here.
\end{solution}

\problem The domain of discourse is the set of integers. Let $S(x, y, z)$ mean that ``$z$ is the sum of $x$ and $y$.''

\subproblem Write a formula that means $x$ is an even integer. 
\subproblem Write a formula that symbolizes the commutative property for addition $(x+y = y+x)$ of integers.
\subproblem Write a formula that symbolizes the associative law for addition of integers:\\ $x + (y + z) = (x + y) + z$.

\begin{solution}
  % Write your answer here.
\end{solution}

\end{document}
