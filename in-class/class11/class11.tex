\documentclass[solution, letterpaper]{cs20inclass}
\usepackage{enumerate}
\usepackage{tikz}
\usepackage{pgf}
\usepackage{tikz}
\usepackage{hyperref}
\usepackage{circuitikz}
\begin{document}
\header{11}{Friday, February 19, 2016}

\noindent Author: Tom Silver% \\

\paragraph*{Midterm Review}
This in-class midterm review is a puzzle. Each problem gives you one piece of the puzzle. Solve all the problems and put together the pieces to discover the keyword. Enjoy!

\problem What follows is alleged to be a proof of the ``Anti-Friendship Theorem'': that for all $n$, there exists a set of $n$ people for which all subsets of 3 people are neither all friends nor all enemies. Find the \textit{first} line that contains a \textbf{logical flaw}, if one exists. (If something is ``unnecessary'' but logically true, it is not a logical flaw.)

\begin{enumerate}
\item Proceeding by induction, we define the predicate $P(n) = $ there exists a set of $n$ people or which all subsets of 3 people are neither all friends nor all enemies.
\item Base cases: Suppose Alice and Bob are friends, Bob and Charlotte are friends, but Alice and Charlotte are not friends. $P(3)$ evidently holds. Add Dianne, a friend of Bob but not of Alice or Charlotte. Thus $P(4)$ holds.
\item Add Eric, a friend of Alice and Charlotte but not Bob or Dianne. Thus $P(5)$ holds as well.
\item Inductive step: Assume for that $n \ge 5$, $P(3), P(4), ..., P(n)$ holds.
\item Let $S$ be a set of $n+1$ people. We can split $S$ into two smaller sets, one with $n$ people and another with 1 person. Call the 1 lonely person $s_0 \in S$ and the set of $n$ people $S_n$.
\item By the inductive hypothesis, any 3 people in $S_n$ are neither all friends nor all enemies.
\item For a subset of 3 in $S_n$, we can take those 3 and add $s_0$ to create a set with 4 people.
\item We showed in the base case that $P(4)$ holds. Therefore this new set of 4 people must contain a subset of 3 who are neither all friends nor all enemies.
\item These 4 people were all from the original set of size $n+1$, so we have found the set we're looking for to prove $P(n+1)$. By induction, the Anti-Friendship Theorem holds.
\item No flaw exists.
\end{enumerate}

\begin{table}[h]
\centering
\begin{tabular}{llllllllll}
1 & 2 & 3 & 4 & 5 & 6 & 7 & 8 & 9 & 10 \\
G & R & S & A & H & L & M & E & O & B 
\end{tabular}
\caption{Problem 1 Clue}
\end{table}

\begin{solution}
  % Write your answer here.
\end{solution}

\pagebreak

\problem Construct a truth table for $(p \leftrightarrow q) \to (q \leftrightarrow r)$. How many rows in the truth table are True?

\begin{table}[h]
\centering
\begin{tabular}{llllllllll}
0 & 1 & 2 & 3 & 4 & 5 & 6 & 7 & 8 & $>8$ \\
E & N & A & V & B & R & I & Z & X & Q            
\end{tabular}
\caption{Problem 2 Clue}
\end{table}

\begin{solution}
  % Write your answer here.
\end{solution}

\problem 

\begin{circuitikz} \draw
(0,2) node[and port] (myand) {}
(2,1) node[or port] (myor) {}
(myand.in 1) node[above left=.5cm](a) {A}
(myand.in 2) node[below left = .5cm](b) {B}
(myand.out) -| (myor.in 1)
(a) -| (myand.in 1)
(b) -| (myand.in 2)
(b) node[below=1cm](c){C}
(c) -| (myor.in 2);
\end{circuitikz}

Which of the following does the above logic circuit compute?

\begin{enumerate}
\item $A \cdot B + C$
\item $A + B + C$
\item $A + B \cdot C$
\item $A\cdot B\cdot C$
\item $(A+ B) \cdot C$
\end{enumerate}

\begin{table}[h]
\centering
\begin{tabular}{llllllll}
1 & 2 & 3 & 4 & 5 \\
S & E & O & T & R
\end{tabular}
\caption{Problem 3 Clue}
\end{table}

\begin{solution}
  % Write your answer here.
\end{solution}

\problem Let $P_1, ...., P_8$ be 8 propositions. Use the quantificational logic statements below to deduce which of the propositions is True.

\begin{itemize}
\item $\forall i, (i < 6 \wedge P_i) \to \neg P_{i-1}$.
\item $\forall i, j, (j > 1 \wedge \neg P_i) \to \neg P_{ij}$ (where $ij$ is multiplication of integers)
\end{itemize}

\begin{table}[h]
\centering
\begin{tabular}{llllllll}
1 & 2 & 3 & 4 & 5 & 6 & 7 & 8 \\
M & N & R & E & P & O & L & A
\end{tabular}
\caption{Problem 4 Clue}
\end{table}

\begin{solution}
  % Write your answer here.
\end{solution}

\problem What is the least value of $m$ for which the following is true? ``In any set of $m$ propositions, all involving only $p$, two of the propositions are logically equivalent.''

\begin{table}[h]
\centering
\begin{tabular}{llllllll}
2 & 5 & 9 & 12 & 17 & 26 & 50 & >50 \\
A & W & B & K & U & O & N & T
\end{tabular}
\caption{Problem 5 Clue}
\end{table}

\begin{solution}
  % Write your answer here.
\end{solution}

\problem \textbf{Final Answer: }

\begin{solution}
  % Write your answer here.
\end{solution}

\end{document}
