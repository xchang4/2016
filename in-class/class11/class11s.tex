\documentclass[solution, letterpaper]{cs20inclass}
\usepackage{enumerate}
\usepackage{tikz}
\usepackage{pgf}
\usepackage{tikz}
\usepackage{hyperref}
\usepackage{circuitikz}
\usepackage{caption}
\begin{document}
\header{11}{Friday, February 19, 2016}

\noindent Author: Tom Silver% \\

\paragraph*{Midterm Review}
This in-class midterm review is a puzzle. Each problem gives you one piece of the puzzle. Solve all the problems and put together the pieces to discover the keyword. Enjoy!

\problem How many base cases does the following proof by induction require?
\\
\\ The Tribonacci numbers are defined by $T_0 = 1, T_1 = 1, T_2 = 2$, and $T_n = T_{n-1} + T_{n-2} + T_{n-3}$ for all $n \ge 3$. The beginning of the Tribonacci sequence is $1, 1, 2, 4, 7, 13, ...$. 
\\
\\ \textbf{Proof.} Let $P(n)$ be the predicate $T_n \le 3^n$. Base cases: [???]. Inductive step: assume $P(1), ..., P(n)$ holds for $n \ge [???]$. Then 
\begin{math}
T_{n+1} = T_n + T_{n-1} + T_{n-2}
\\ \le 3^n + 3^{n-1} + 3^{n-2}
\\ = 3^{n+1}(\frac{1}{3} + \frac{1}{9} + \frac{1}{27})
\\ = 3^{n+1}\frac{13}{27}
\\ \le 3^{k+1}
\end{math} 


\begin{table}[h]
\centering
\begin{tabular}{lllllll}
0 & 1 & 2 & 3 & 4 & 5 & $>5$ \\
G & R & S & E & H & L & M
\end{tabular}
\caption*{Problem 1 Clue}
\end{table}

\begin{solution}
3 base cases are required.
\end{solution}

\problem Construct a truth table for $(p \leftrightarrow q) \to (q \leftrightarrow r)$. How many rows in the truth table are True?

\begin{table}[h]
\centering
\begin{tabular}{llllllllll}
0 & 1 & 2 & 3 & 4 & 5 & 6 & 7 & 8 & $>8$ \\
E & N & A & V & B & R & I & Z & X & Q            
\end{tabular}
\caption*{Problem 2 Clue}
\end{table}

\begin{solution}

6 rows are True:

\begin{table}[h]
\centering
\begin{tabular}{|l|l|l|l|}
\hline
$p$ &$ q$ & $r$ & $(p \leftrightarrow q) \to (q \leftrightarrow r)$ \\ \hline
1 & 1 & 1 & 1                                               \\ \hline
1 & 1 & 0 & 0                                               \\ \hline
1 & 0 & 1 & 1                                               \\ \hline
1 & 0 & 0 & 1                                               \\ \hline
0 & 1 & 1 & 1                                               \\ \hline
0 & 1 & 0 & 1                                               \\ \hline
0 & 0 & 1 & 0                                               \\ \hline
0 & 0 & 0 & 1                                               \\ \hline
\end{tabular}
\end{table}
\end{solution}

\problem 

\begin{circuitikz} \draw
(0,2) node[and port] (myand) {}
(2,1) node[or port] (myor) {}
(myand.in 1) node[above left=.5cm](a) {A}
(myand.in 2) node[below left = .5cm](b) {B}
(myand.out) -| (myor.in 1)
(a) -| (myand.in 1)
(b) -| (myand.in 2)
(b) node[below=1cm](c){C}
(c) -| (myor.in 2);
\end{circuitikz}

Which of the following does the above logic circuit compute? (For the purpose of this problem, assume $1+1 = 1$.)

\begin{enumerate}
\item $A \cdot B + C$
\item $A + B + C$
\item $A + B \cdot C$
\item $A\cdot B\cdot C$
\item $(A+ B) \cdot C$
\end{enumerate}

\begin{table}[h]
\centering
\begin{tabular}{llllllll}
1 & 2 & 3 & 4 & 5 \\
S & E & O & T & R
\end{tabular}
\caption*{Problem 3 Clue}
\end{table}

\begin{solution}
(1).
\end{solution}

\problem Which of the following quantificational logic statements are true?

\begin{enumerate}
\item $\forall n \in \mathbb{N}, \exists m \in \mathbb{N} . n\cdot m = 1$
\item $\forall n \in \mathbb{N}, \forall m \in \mathbb{N} . n+m = n$
\item $\exists n \in \mathbb{N}, \forall m \in \mathbb{N} . n+m = n$
\item $\exists n \in \mathbb{N}, \forall m \in \mathbb{N} . n\cdot m = n+m$
\item $\exists n \in \mathbb{N}, \forall m \in \mathbb{N} . n\cdot m = m$
\end{enumerate}

\begin{table}[h]
\centering
\begin{tabular}{llllllll}
1 & 2 & 3 & 4 & 5 \\
M & N & R & E & L
\end{tabular}
\caption*{Problem 4 Clue}
\end{table}

\begin{solution}
(5) is true; $n = 1$.
\end{solution}

\problem (BONUS) What is the least value of $m$ for which the following is true? ``In any set of $m$ propositions, all involving only $p$, two of the propositions are logically equivalent.''

\begin{table}[h]
\centering
\begin{tabular}{llllllll}
2 & 5 & 9 & 12 & 17 & 26 & 50 & $>50$ \\
A & W & B & K & U & O & N & T
\end{tabular}
\caption*{Problem 5 Clue}
\end{table}

\begin{solution}
By the pigeonhole principle, 5, since there are only $4$ possible truth tables.
\end{solution}

\problem \textbf{Final Answer: }

\begin{solution}
LEWIS
\end{solution}

\end{document}
