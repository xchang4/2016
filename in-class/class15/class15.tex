\documentclass[solution, letterpaper]{cs20inclass}
\usepackage{enumerate}
\usepackage{tikz}
\usepackage{pgf}
\usepackage{tikz}
\usepackage{hyperref}
\usepackage{ dsfont }
\usepackage{amsmath}
\begin{document}
\header{15}{Wednesday, March 02, 2016}

\noindent Author: Crystal Chang

\paragraph*{Executive Summary}
\begin{enumerate}
\item \textbf{Definition of Recursive Data Types:} common way of defining mathematical objects, which says how to construct new data elements from previous ones.
\begin{itemize}
\item {\em Base Case(s):} specify that some known mathematical elements are in the data type
\item {\em Constructor Rule(s):} specify how to construct new data elements from previously constructed elements or from base elements.
\item {\em “Nothing else” (generally implicit):} the only way you can get whatever is you defining is by starting from the base case(s) and applying the constructor rule(s) one or more times.
\end{itemize}

\item \textbf{The Principle of Structural Induction:} to prove P(x) holds for all x in a recursively defined set S, prove
\begin{itemize}
\item {\em Basis Step:} P(b) is true for each base case element $b \in S$, and 
\item {\em Recursive Step:} P(c($x_1$,...,$x_k$)) for each constructor c, assuming as the induction hypothesis that P($x_1$),..., and P($x_k$) all hold.
\end{itemize}
\end{enumerate}

%% PROBLEM 1 %%
\problem Recursive Definition:
\subproblem There's an error in the following definition of the set of even integers (EI). Find the error and fix it. 
\begin{itemize}
\item Base Case:  $0\in EI$
\item Constructor Rule: For any element x in EI, x+2 is in EI. 
\item “Nothing else” (generally implicit): Nothing is in NE unless it is obtained from the base case and constructor rule.
\end{itemize}
\subproblem Give a recursive definition of the natural numbers $\mathbb{N}$.
\subproblem Give a recursive definition of the sequence $b_n$, $b_n=2n+5,  n\in\mathbb{N}$

%% PROBLEM 2 %%
\problem Let S be the set defined as follows:
\begin{itemize}
\item Base Case: $(1,2) \in S$
\item Constructor Rules: If $(x,y)\in S$, then C1: $(x+2, y) \in S$, C2: $(y,x) \in S$ 
\end{itemize}
\subproblem Is $(4,3)\in S$? If it is, how can you derive it from (1,2)?
\subproblem Use induction to prove that $(2n+2, 2n+1)\in S$ for all n $\in \mathbb{N}$.
\pagebreak

%% PROBLEM 3 %%
\problem Let S be the set defined as follows:
\begin{itemize}
\item Base Case: $(0,0) \in S$
\item Constructor Rules: If $(a,b)\in S$, then C1: $(a, b+1) \in S$, C2: $(a+1,b+1)\in S$ and C3: $(a+2, b+1)\in S$ 
\end{itemize}
\subproblem List 5 elements in set S.
\subproblem Use structural induction to prove that for every $(a,b)\in S, a\leq 2b$.


%% PROBLEM 4%%
\problem (Bonus) Construct a recursive definition for the set of strings \textbf{S} over the alphabet a,b excepting empty string, i.e. set of string consisting of a's and b's such as abbab, bbabaa, etc.

\end{document}