\documentclass[solution, letterpaper]{cs20inclass}
\usepackage{enumerate}
\usepackage{tikz}
\usepackage{pgf}
\usepackage{tikz}
\usepackage{hyperref}
\usepackage{ dsfont }
\usepackage{amsmath}
\begin{document}
\header{15}{Wednesday, March 02, 2016}

\noindent Author: Crystal Chang

\paragraph*{Executive Summary}
\begin{enumerate}
\item \textbf{Definition of Recursive Data Types:} common way of defining mathematical objects, which says how to construct new data elements from previous ones.
\begin{itemize}
\item {\em Base Case(s):} specify that some known mathematical elements are in the data type
\item {\em Constructor Rule(s):} specify how to construct new data elements from previously constructed elements or from base elements.
\item {\em “Nothing else” (generally implicit):} the only way you can get whatever is you defining is by starting from the base case(s) and applying the constructor rule(s) one or more times.
\end{itemize}

\item \textbf{The Principle of Structural Induction:} to prove P(x) holds for all x in a recursively defined set S, prove
\begin{itemize}
\item {\em Basis Step:} P(b) is true for each base case element $b \in S$, and 
\item {\em Recursive Step:} P(c($x_1$,...,$x_k$)) for each constructor c, assuming as the induction hypothesis that P($x_1$),..., and P($x_k$) all hold.
\end{itemize}
\end{enumerate}

%% PROBLEM 1 %%
\problem Recursive Definition:
\subproblem There's an error in the following definition of the set of even integers (E). Find the error and fix it. 
\begin{itemize}
\item Base Case:  $0\in E$
\item Constructor Rule: For any element x in E, x+2 is in E. 
\item “Nothing else” (generally implicit): Nothing is in E unless it is obtained from the base case and constructor rule.
\end{itemize}
\subproblem Give a recursive definition of the natural numbers $\mathbb{N}$.
\subproblem Give a recursive definition of the sequence $b_n$, $b_n=2n+5,  n\in\mathbb{N}$
\pagebreak

\begin{solution}
\subsolution  It doesn't include negative Even Integer. There should be one more constructor rule “x-2 is in EI”.
\subsolution 
\begin{itemize}
\item Base Cases:  $1\in \mathbb{N}$
\item Constructor Rule: If $n\in\mathbb{N}$, then $n+1\in\mathbb{N}$.
\item “Nothing else” (generally implicit): Nothing is in $\mathbb{N}$ unless it is obtained from the base case and constructor rule.
\end{itemize}
\subsolution
\begin{itemize}
\item Base Cases:  $b_1=7$
\item Constructor Rule: $b_{n+1}=b_n+2$ for $n\in\mathbb{N}$
\item “Nothing else” (generally implicit): Nothing is in $b_n$ unless it is obtained from the base case and constructor rule.
\end{itemize}
\end{solution}

%% PROBLEM 2 %%
\problem Let S be the set defined as follows:
\begin{itemize}
\item Base Case: $(1,2) \in S$
\item Constructor Rules: If $(x,y)\in S$, then C1: $(x+2, y) \in S$ and C2: $(y,x) \in S$ 
\end{itemize}
\subproblem Is $(4,3)\in S$? If it is, how can you derive it from (1,2)?
\subproblem Use induction to prove that $(2n+2, 2n+1)\in S$ for all n $\in \mathbb{N}$.

\begin{solution}
\subsolution Yes, it is. Apply C1 to (1,2), we can get (3,2); Apply C2 to (3,2), we can get (2,3); Apply C1 to (2,3), we can get (4,3).
\subsolution Let P(n): $(2n+2, 2n+1)\in S$. We must show that for all $n\in\mathbb{N}, P(n)$.
\begin{itemize}
\item Base case: When n=1, $(4,3)\in S$. (Already proved in (A))
\item Induction step: \\
Assuming P(n): $(2n+2, 2n+1)\in S$ holds, we want to prove that $P(n+1): (2(n+1)+2, 2(n+1)+1)=(2n+4, 2n+3)\in S$.\\
Apply C1 to (2n+2, 2n+1), we then could get $(2n+4, 2n+1)\in S$. Then apply C2 to (2n+4, 2n+1), we could get $(2n+1, 2n+4)\in S$. Then, apply C1 to (2n+1, 2n+4), then we can get $(2n+3, 2n+4)\in S$. Finally, apply C2 again to (2n+3, 2n+4), then we can get $(2n+4, 2n+3)\in S$.
\end{itemize}
\end{solution}

%% PROBLEM 3%%
\problem (Bonus) A palindrome is a sequence of characters (do not need to be a word) which reads the same backward or forward e.g. dad, mom, abba, moom... Let's define the set $\sum$ as the set of all letters $\{a,b,c,...,z\}$, $\lambda$ as the empty string and P as the set of all palindromes(excepting empty string). Give a recursive definition for the set P.

\begin{solution}
\begin{itemize}
\item Base Cases: $\forall x\in \sum, x\in P$
\item Constructor Rule:If $p \in P$,  $\forall x\in \sum$, $xpx\in P$
\item “Nothing else” (generally implicit): Nothing is in P unless it is obtained from the base case and constructor rule.
\end{itemize}
\end{solution}

%% PROBLEM 4 %%
\problem (Bonus) Let S be the set defined as follows:
\begin{itemize}
\item Base Case: $(0,0) \in S$
\item Constructor Rules: If $(a,b)\in S$, then C1: $(a, b+1) \in S$, C2: $(a+1,b+1)\in S$ and C3: $(a+2, b+1)\in S$ 
\end{itemize}
\subproblem List 5 elements.
\subproblem Use structural induction to prove that for every $(a,b)\in S, a\leq 2b$.

\begin{solution}
\subsolution (0,1), (1,1), (2,1), (1,2), (2,2)...
\subsolution
\begin{itemize}
\item Basis step: By the base case of the definition of S, $(0,0)\in S$. $0\leq2(0)$.
\item Recursive step: \\
Now consider the constructor rule in the definition of S. Assume elements $a,b\in S$ and $ a\leq 2b$. We must  show that $a\leq2(b+1)$, $(a+1)\leq2(b+1)$ and $(a+2)\leq2(b+1)$.
\subitem 1. prove $a\leq2(b+1)$:  $a\leq2b\rightarrow a\leq(2b+2)=2(b+1)$
\subitem 2. prove $(a+1)\leq2(b+1)$: $a\leq2b \rightarrow (a+1)\leq(2b+1)\leq(2b+2)=2(b+1)$
\subitem 3. prove $(a+2)\leq2(b+1)$:  $a\leq2b \rightarrow (a+2)\leq(2b+2)=2(b+1)$
\end{itemize}
\end{solution}

\end{document}