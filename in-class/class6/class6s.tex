\documentclass[solution, letterpaper]{cs20inclass}
\usepackage{enumerate}
\usepackage{tikz}
\usepackage{pgf}
\usepackage{tikz}
\usepackage{hyperref}
\begin{document}
\header{6}{Friday, February 5, 2016}

\noindent Author: Tom Silver% \\

\paragraph*{Executive Summary}
\begin{enumerate}

\item  A proposition is a statement that is either \emph{True} or \emph{False}. For brevity, we label propositions with letters ($p$ : ``Gold is a metal'').  In this case we can also say that \emph{p} is a propositional (Boolean) variable. 

\item Logical operators combine propositions into \emph{compound} propositions . List of the most common logical operators: 


\begin{tabular}{ | l l l |}
\hline
Symbol & Operation & Note\\
\hline
$\land$ & AND (Conjunction) & $p \land q$ is true when both $p$ and $q$ are true.\\
\hline
$\lor$ & OR (Disjunction)& $p \lor q$ is true when at least one of $p$ and $q$ is true.\\
\hline
$\neg$ & Negation & Another symbol is the horizontal bar: $\neg(p \land q) \equiv \overline{p \land q}$\\
\hline
$\oplus$ & Exclusive OR. & Like OR but it's false when both $p$ and $q$ are true. \\
\hline
$\to$ & Implication & Important equivalence: $p\to q \equiv \neg p \lor q$\\
\hline
$\leftrightarrow$ & IFF & Important equivalence: $p \leftrightarrow q \equiv (p \to q) \land (q \to p)$\\
\hline
\end{tabular}
\end{enumerate}

\problem

Prove by truth table the first of the two distributive laws: 
\begin{center}
$p \lor (q \land r) \equiv (p \lor q) \land (p \lor r)$
\end{center}

\begin{solution}

\begin{table}[h]
\centering
\label{my-label}
\begin{tabular}{lllll}
$p$ & $q$ & $r$ & $p \lor (q \land r)$ & $(p \lor q) \land (p \lor r)$ \\
1 & 1 & 1 & 1                    & 1                           \\
1 & 1 & 0 & 1                    & 1                           \\
1 & 0 & 1 & 1                    & 1                           \\
1 & 0 & 0 & 1                    & 1                           \\
0 & 1 & 1 & 1                    & 1                           \\
0 & 1 & 0 & 0                    & 0                           \\
0 & 0 & 1 & 0                    & 0                           \\
0 & 0 & 0 & 0                    & 0                          
\end{tabular}
\end{table}

\end{solution}

\problem

\subproblem	
Using the propositions p=``I study'', q=``I will pass the course'', r=``The professor accepts bribes'',  translate the following into statements of propositional logic: 
	\begin{enumerate}
	\item If I do not study, then I will not pass the course unless the professor accepts bribes.
	\item If the professor accepts bribes, then I will pass the course without studying.
	\item The professor does not accept bribes, but I study and will pass the course.
	\end{enumerate}
	
	
\subproblem Using the propositions p=``The night hunting is successful'', q=``The moon is full'', r=``The sky is cloudless'', translate the following into statements of propositional logic: 
	
	\begin{enumerate}
	\item For successful night hunting it is necessary that the moon is full and the sky is cloudless.
	\item The sky being cloudy is both necessary and sufficient for the night hunting to be successful.
	\item  If the sky is cloudy, then the night hunting will not be successful unless the moon is full.
	\end{enumerate}

\begin{solution}

\subsolution

\begin{enumerate}
\item $(\neg p \land q) \to r$
\item $(r \land \neg p) \to q$
\item $\neg r \land p \land q$
\end{enumerate}

\subsolution

\begin{enumerate}
\item $p \to (q \land r)$
\item $\neg r \leftrightarrow q$
\item $(\neg r \land p) \to q$
\end{enumerate}

\end{solution}

\problem (BONUS) In the mini-lecture you saw how to add two boolean variables using logic operators. In this problem we will show how to subtract two boolean variables.\\
Consider the boolean variables $a, b, c$ each representing a single bit. Write two propositional formulas: one for the result of the boolean subtraction $a-b$, and another for the the ``borrow bit" $c$, indicating whether a bit must be borrowed from an imaginary bit to the left of $a$ (the case when a=0 and b=1) to ensure that the result of the subtraction is always positive. Start by creating a ``truth'' table for $a-b$ and the borrow bit $c$. The borrow bit $c$ is always 1 initially, and is set to 0 only if borrowing was necessary. 

\begin{solution}

\begin{table}[]
\centering
\caption{My caption}
\label{my-label}
\begin{tabular}{llll}
$a$ & $b$ & $a-b$ & $c$ \\
1 & 1 & 0   & 0 \\
1 & 0 & 1   & 0 \\
0 & 1 & 1   & 1 \\
0 & 0 & 0   & 0
\end{tabular}
\end{table}

So: $a-b = a \oplus b$ and $c = \neg a \land b$.
 
\end{solution}

\problem (BONUS) Using a truth table, find which of the following compound propositions are always true, regardless of the values of p and q: 

\begin{enumerate} 
\item $p \rightarrow (p \lor q)$
\item $\lnot (p \rightarrow (p \lor q))$
\item $p \rightarrow (p \rightarrow q)$
\end{enumerate}

\begin{solution}

\begin{enumerate} 
\item Always true
\item Always false
\item Not true for $p = 1, q = 0$
\end{enumerate}

\end{solution}

\end{document}
