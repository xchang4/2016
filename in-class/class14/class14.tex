\documentclass[solution, letterpaper]{cs20inclass}
\usepackage{enumerate}
\usepackage{tikz}
\usepackage{pgf}
\usepackage{tikz}
\usepackage{hyperref}
\begin{document}
\header{14}{Monday, February 29, 2016}

\noindent Author: Ben Zheng

\paragraph*{Executive Summary}
\begin{enumerate}

\item \textbf{Some set notation}

Given a set $S = \{0,1\}$, we have that:
\begin{itemize}
\item $\{0,1\}^n$ is the set of strings of exactly length $n$: e.g. $01001 \in \{0,1\}^5.$
\item $\{0,1\}^*$ is the set of strings of finite length, including the empty string: 

e.g. $010010001 \in \{0,1\}^*.$
\item $\{0,1\}^{\omega}$ is the set of sequences of infinite length: e.g. $010010001\cdots.$ 

NOTE: We say ``sequence'' because strings are defined to have finite length 

(i.e. they are finite sequences).
\item The collection of all subsets of $S$ is its power set, denoted $\mathcal{P}(S)$. Note that $\emptyset \in \mathcal{P}(S)$ for all $S$.
\end{itemize}

\item \textbf{Countable sets}
\begin{itemize}
\item Two finite sets $A$ and $B$ have the same cardinality if there is a bijection between them: i.e. $A$ bij $B$.
\item An infinite set $A$ is called \emph{countably infinite} if $A$ bij $\mathbb{N}$.
\item The set of all integers $\mathbb{Z}$ is countably infinite.
\item For finite sets $A$ and $B$, $A$ is a proper subset of $B$ if $A \subseteq B$ and $|A| < |B|.$ For countably infinite sets this is not necessarily so!
\item Countably infinite sets are closed under the following operations: subset, intersection, Cartesian product and countably infinite union.
\item We use ``countable'' to refer to sets that are finite or countably infinite.
\end{itemize}

\item \textbf{Uncountable sets}
\begin{itemize}
\item {\bf Cantor's Theorem:} For any set $A$, the cardinality of $\mathcal{P}(A)$ is greater than that of $A$, i.e. a bijection $f$ does not exist between $A$ and $\mathcal{P}(A)$.
\item Proof approach: Given a bijection $f$, consider the set $W$ consisting of elements in $A$ that are matched to elements in $\mathcal{P}(A)$ that do not contain them (remember, an element in $\mathcal{P}(A)$ is a subset!). By the definition of $f$, some element in $A$ must match to $W$ since $W$ is a subset of $A$ and thus an element of $\mathcal{P}(A)$, but by the definition of $W$ no element in $A$ can match to $W$, which is a contradiction.
\item Uncountable sets: $S^\omega$ for any set $S$ such that $|S|>1$, $\mathcal{P}(\mathbb{N})$, and the set of real numbers within any interval.
\end{itemize}

\end{enumerate}
\pagebreak

%% Problem 1
\problem Suppose $S=\{0,1\}^*$. Which of the following sets are countable?
\subproblem The union of two finite sets
\subproblem The powerset of a countably infinite set
\subproblem The union of a finite set and a countably infinite set
\subproblem The powerset of a finite set
\subproblem $\bigcup_{i\geq 0} S_i$, where $S_i=\{s\ | \ s \in S, \ |s|=i\}$
\subproblem $S \times S$
\subproblem The set of all functions from $\mathbb{N}$ to $\mathbb{N}$

%% Problem 2
\problem Given the finite set $S$ with $n$ unique elements, what is the cardinality of $\mathcal{P}(S)$?

%% Problem 3
\problem Show that the difference of an uncountable set and a countable set is uncountable.

%% Problem 4
\problem (BONUS) Show that the Cartesian product $\mathbb{N} \times \mathbb{N} = \{(a,b)|a,b \in \mathbb{N}\}$ is countably infinite by creating a bijection between $\mathbb{N}$ and $\mathbb{N} \times \mathbb{N}$.

\end{document}