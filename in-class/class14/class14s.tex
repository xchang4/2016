{\rtf1\ansi\ansicpg1252\deff0\deflang1033{\fonttbl{\f0\fnil\fcharset0 Calibri;}}
{\*\generator Msftedit 5.41.21.2510;}\viewkind4\uc1\pard\sa200\sl276\slmult1\lang9\f0\fs22\\documentclass[solution, letterpaper]\{cs20inclass\}\par
\\usepackage\{enumerate\}\par
\\usepackage\{tikz\}\par
\\usepackage\{pgf\}\par
\\usepackage\{tikz\}\par
\\usepackage\{hyperref\}\par
\\begin\{document\}\par
\\header\{14\}\{Monday, February 29, 2016\}\par
\par
\\noindent Author: Ben Zheng\par
\par
\\paragraph*\{Executive Summary\}\par
\\begin\{enumerate\}\par
\par
\\item \\textbf\{Some set notation\}\par
\par
Given a set $S = \\\{0,1\\\}$, we have that:\par
\\begin\{itemize\}\par
\\item $\\\{0,1\\\}^n$ is the set of strings of exactly length $n$: e.g. $01001 \\in \\\{0,1\\\}^5.$\par
\\item $\\\{0,1\\\}^*$ is the set of strings of finite length, including the empty string: \par
\par
e.g. $010010001 \\in \\\{0,1\\\}^*.$\par
\\item $\\\{0,1\\\}^\{\\omega\}$ is the set of sequences of infinite length: e.g. $010010001\\cdots.$ \par
\par
NOTE: We say ``sequence'' because strings are defined to have finite length \par
\par
(i.e. they are finite sequences).\par
\\item The collection of all subsets of $S$ is its power set, denoted $\\mathcal\{P\}(S)$. Note that $\\emptyset \\in \\mathcal\{P\}(S)$ for all $S$.\par
\\end\{itemize\}\par
\par
\\item \\textbf\{Countable sets\}\par
\\begin\{itemize\}\par
\\item Two finite sets $A$ and $B$ have the same cardinality if there is a bijection between them: i.e. $A$ bij $B$.\par
\\item An infinite set $A$ is called \\emph\{countably infinite\} if $A$ bij $\\mathbb\{N\}$.\par
\\item The set of all integers $\\mathbb\{Z\}$ is countably infinite.\par
\\item For finite sets $A$ and $B$, $A$ is a proper subset of $B$ if $A \\subseteq B$ and $|A| < |B|.$ For countably infinite sets this is not necessarily so!\par
\\item Countably infinite sets are closed under the following operations: subset, intersection, Cartesian product and countably infinite union.\par
\\item We use ``countable'' to refer to sets that are finite or countably infinite.\par
\\end\{itemize\}\par
\par
\\item \\textbf\{Uncountable sets\}\par
\\begin\{itemize\}\par
\\item \{\\bf Cantor's Theorem:\} For any set $A$, the cardinality of $\\mathcal\{P\}(A)$ is greater than that of $A$, i.e. a bijection $f$ does not exist between $A$ and $\\mathcal\{P\}(A)$.\par
\\item Proof approach: Given a bijection $f$, consider the set $W$ consisting of elements in $A$ that are matched to elements in $\\mathcal\{P\}(A)$ that do not contain them (remember, an element in $\\mathcal\{P\}(A)$ is a subset!). By the definition of $f$, some element in $A$ must match to $W$ since $W$ is a subset of $A$ and thus an element of $\\mathcal\{P\}(A)$, but by the definition of $W$ no element in $A$ can match to $W$, which is a contradiction.\par
\\item Uncountable sets: $S^\\omega$ for any set $S$ such that $|S|>1$, $\\mathcal\{P\}(\\mathbb\{N\})$, and the set of real numbers within any interval.\par
\\end\{itemize\}\par
\par
\\end\{enumerate\}\par
\\pagebreak\par
\par
%% Problem 1\par
\\problem Suppose $S=\\\{0,1\\\}^*$. Which of the following sets are countable?\par
\\subproblem The union of two finite sets\par
\\subproblem The powerset of a countably infinite set\par
\\subproblem The union of a finite set and a countably infinite set\par
\\subproblem The powerset of a finite set\par
\\subproblem $\\bigcup_\{i\\geq 0\} S_i$, where $S_i=\\\{s\\ | \\ s \\in S, \\ |s|=i\\\}$\par
\\subproblem $S \\times S$\par
\\subproblem The set of all functions from $\\mathbb\{N\}$ to $\\mathbb\{N\}$\par
\par
\par
\\begin\{solution\}\par
A and C are countable since countable sets are closed under union with a finite number of countable sets. B is uncountable by definition. D is countable since it is a finite set. E is equivalent to $S$, which is countable. F is countable since we know that countable sets are closed under Cartesian product. G is uncountable using Cantor's diagonlization argument.\par
\\end\{solution\}\par
\par
%% Problem 2\par
\\problem Given the finite set $S$ with $n$ unique elements, what is the cardinality of $\\mathcal\{P\}(S)$?\par
\par
\\begin\{solution\}\par
For every subset of $S$ in $\\mathcal\{P\}(S)$, each element in $S$ is either included in the subset or not included. This gives us a total of \\fbox\{$2^n$\} unique subsets that are elements in $\\mathcal\{P\}(S)$.\par
\\end\{solution\}\par
\par
%% Problem 3\par
\\problem Show that the difference of an uncountable set and a countable set is uncountable.\par
\par
\\begin\{solution\}\par
We perform a proof by contradiction. For an arbitrary uncountable set $A$ and an arbitrary\par
countable set $B$, we assume first that $A - B$ is countable. Next, we note that $A - B = A - A \\cap B$. But $A \\cap B$ is countable since $|A \\cap B| \\leq |B|$ and $B$ is countable. Since the union of countable sets is countable, this implies that $(A - A \\cap B) \\cup (A \\cap B) = A$ is countable as well, a contradiction.\par
\\end\{solution\}\par
\par
%% Problem 4\par
\\problem (BONUS) Show that the Cartesian product $\\mathbb\{N\} \\times \\mathbb\{N\} = \\\{(a,b)|a,b \\in \\mathbb\{N\}\\\}$ is countably infinite by creating a bijection between $\\mathbb\{N\}$ and $\\mathbb\{N\} \\times \\mathbb\{N\}$.\par
\par
\\begin\{solution\}\par
To create a bijection $f$ we first note that for any positive integer $n$, there are exactly $n$ elements $(a,b)$ in $\\mathbb\{N\} \\times \\mathbb\{N\} $ such that $a + b = n + 1$. Hence, we can map the first natural number $1$ to elements in $\\mathbb\{N\} \\times \\mathbb\{N\} $ whose components sum to $2$ (in this case $(1,1)$), the next two natural numbers to elements whose components sum to $3$ in order from smallest-to-largest first component ($(1,2)$ and $(2,1)$), and so on. We know that $f$ is surjective since every element in $\\mathbb\{N\} \\times \\mathbb\{N\} $ can be mapped to in this way, and $f$ is injective because for every positive integer $n$ we can map $n$ unique elements in $\\mathbb\{N\}$ to $n$ distinct elements $(a,b)$ in $\\mathbb\{N\} \\times \\mathbb\{N\} $ where $a + b = n + 1$. Therefore, $f$ is a bijection.\par
\\end\{solution\}\par
\par
\\end\{document\}\par
}
