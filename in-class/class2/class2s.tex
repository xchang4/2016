%% Please fill in your name and collaboration statement here.
\newcommand{\studentName}{**FILL IN YOUR NAME HERE**}
\newcommand{\collaborationStatement}{**FILL IN YOUR COLLABORATION STATEMENT HERE \\ (See the syllabus for information)**}


%%%%%%%%%%%%%%%%%%%%%%%%%%%%%%%%%%%%%%%%%%%%%%%
\documentclass[solution, letterpaper]{cs20inclass}
\usepackage{enumerate}
\usepackage{tikz}
\usepackage{pgf}
\usepackage{tikz}
\usepackage{hyperref}
\begin{document}
\header{2}{Wednesday, January 27, 2016}

\noindent Author: Hannah Blumberg% \\

\paragraph*{Check-In Questions}
\begin{enumerate}
\item Which of the following statements is equivalent to "if A, then B"?
\begin{enumerate}
\item	if not B then not A
\item	A if B
\item	if not A then not B
\item	if not B then A
\end{enumerate}

\textbf{Solution:} (a)

The contrapositive of ``A $\implies$ B'' is ``not B $\implies$ not A.''

\end{enumerate}

\problem

Each of the following claims and corresponding proofs is incorrect. Explain why each proof is incorrect.

\subproblem Claim: $1c = \$1$   \[ 1c = \$0.01= (\$ 0.1)^2 = (10c)^2 = 100c =\$1 \]

\subproblem Claim: If $a=b$ then $a=0$\\ \[a=b\]\[a^2=ab\]\[a^2-b^2=ab-b^2\]\[(a-b)(a+b)=(a-b)b\]\[a+b=b\]\[a=0\]

\begin{solution}
  \subsolution $(10c)^2 = 100c$ is not true. The correct equality is $(10c)^2 = 100c^2$.
  \\
  \subsolution You cannot divide $(a-b)(a+b) = (a-b)b$ by $(a-b)$ since $(a-b)$ might be 0.
\end{solution}

\problem

Prove that in any group of six people, at least two of them know the same number of people. Note that you don't know yourself, and that if A knows B then B knows A (thus ``knows'' is a symmetric relation). 

\begin{solution}

There are 6 possible numbers of people a person can know (0 through 5 since you cannot know yourself).

Suppose that each person knows a different number of people. Then someone (person A) knows 0 people and someone (person B) else knows 5 people. Since B knows 5 people and cannot know him or herself, then B knows A. Since knowing is a symmetric relation, A knows B. This is a contradiction because A knows 0 people.

Since we have arrived at a contradiction, we can conclude that in any group of six people, at least two of them know the same number of people.

\end{solution}

\problem

(BONUS) Prove by contradiction that $\sqrt[3]{4}$ is irrational.

\begin{solution}

Suppose $\sqrt[3]{4}$ is rational. Then there exist two integers $m$ and $n$ such that $\sqrt[3]{4} = \frac{m}{n}$ and $m$ and $n$ are relatively prime.

\begin{center}
$\sqrt[3]{4} = \frac{m}{n}$

$4 = (\frac{m}{n})^3$

$4 = \frac{m^3}{n^3}$

$m^3 = 4n^3$

$\frac{1}{2}m^3 = 2n^3$
\end{center}

Since $n \in \mathbb{Z}$, $2n^3 \in \mathbb{Z}$.

Since $2n^3 \in \mathbb{Z}$ and $\frac{1}{2}m^3 = 2n^3$, $m^3$ must be divisible by 2. 

Since $m^3$ is divisible by 2, $m$ or $m^2$ (the factors of $m^3$) must be divisible by 2.

If $m^2$ is even, then $m$ must also be even.* We can therefore conclude that $m$ is even.

Then there exists some $z \in \mathbb{Z}$ such that $m=2z$.

\begin{center}
$m^3 = 4n^3$

$(2z)^3 = 4n^3$

$8z^3 = 4n^3$

$2z^3 = n^3$

$z^3 = \frac{1}{2}n^3$
\end{center}

Since $z \in \mathbb{Z}$, $z^3 \in \mathbb{Z}$ and $n^3$ is divisible by 2.

Since $n^3$ is divisible by 2, $n$ or $n^2$ (the factors of $n^3$) must be divisible by 2.

If $n^2$ is even, then $n$ must also be even.* We can therefore conclude that $n$ is even.

Since both $m$ and $n$ are even, we have found a contradiction. (Recall that we chose $m$ and $n$ such that $m$ and $n$ are relatively prime.)

We can therefore conclude that $\sqrt[3]{4}$ is irrational.\\

*Lemma: For all $x \in \mathbb{Z}$, $x^2$ is even $\implies$ $x$ is even. 

Consider the contrapositive, $x$ is not even $\implies$ $x^2$ is not even.

If $x$ is not even, there is some $y \in \mathbb{Z}$ such that $x = 2y + 1$.

Then $x^2 = (2y+1)(2y+1) = 4y^2 + 4y + 1 = 2(2y^2+2y)+1$.

Since $2y^2+2y \in \mathbb{Z}$, $x^2$ must not be even.

\end{solution}

\end{document}
