\documentclass[12pt]{article}
\usepackage{latexsym}
\usepackage{amssymb,amsmath}
\usepackage[pdftex]{graphicx}
\usepackage{color}


\topmargin = 0.1in \textwidth=5.7in \textheight=8.6in

\oddsidemargin = 0.2in \evensidemargin = 0.2in

\newcommand{\cabal}{\text{cabal}}
\newcommand{\PersonA}{\text{Charles}}
\newcommand{\PersonB}{\text{Corinne}}
\newcommand{\PersonC}{\text{Hannah}}
\newcommand{\PersonD}{\text{Jenny}}
\newcommand{\PersonE}{\text{Keenan}}
\newcommand{\PersonF}{\text{Tom}}


\begin{document}

\begin{center}
COMPUTER SCIENCE 20, SPRING 2016 \\

\smallskip

Module \#9 (Quantificational Logic I)
\end{center}
Author: Hannah Blumberg, Edited by Michelle Danoff\\
Reviewers: Steve Komarov\\
Last modified: February 12, 2016

\medskip

\paragraph*{Executive Summary}
\begin{enumerate}

\item Propositions and predicates
\begin{itemize}

\item A proposition $P$ is like a boolean variable. Its value is either ``true'' or ``false.''
\item A predicate $P(x)$ is like a boolean-valued function. It may have the value ``true'' for some some set of input values and the value ``false'' for others.
\item A predicate can have more than one argument; e.g. Enrolled($x, y$) might mean ``student $x$ is enrolled in course $y$.'' 
\item Teh values that  $x$ and $y$ might assume are the domain of the function. 
\end{itemize}

\item Quantifiers
\begin{itemize}
\item The existential quantifier: $\exists x P(x)$ or $\exists x. P(x)$ or $\exists x \mbox{ s.t. }P(x)$ means ``there exists at least one $x$ in the domain $D$ such that $P(x)$ is true.

\item The universal quantifier: $\forall x P(x)$ or $\forall x. P(x)$  means ``for every $x$ in the domain $D$, $P(x)$ is true.

\end{itemize}

\item Multiple quantifiers
\begin{itemize}

\item  $\exists x P(x)$ and $\forall x P(x)$ are both propositions, subject to the rules of logic that you already know.

\item  $\exists x P(x,y)$ and $\forall x P(x,y)$ are both predicates of the form $Q(y)$, subject to the rules of quantificational logic that you are learning.

\item $\exists x \exists y P(x,y)$ and $\forall x \forall y P(x,y)$ are both propositions. The order of the quanitifiers is irrelevant.

\item $\exists x \forall y P(x,y)$ and $\forall y \exists x P(x,y)$ are both propositions, but they are not equivalent! The order of the quantifiers is important. 

\end{itemize}

\item Negation and quantifiers
\begin{itemize}
\item $\neg (\exists x. P(x)) \leftrightarrow \forall x. (\neg P(x))$.
\item $\neg (\forall x. P(x)) \leftrightarrow \exists x. (\neg P(x))$.

\end{itemize}

\end{enumerate}

\pagebreak

\paragraph*{Small group problems}

\begin{enumerate}

\item Convert the following English sentences into logical formulas. You may use expressions like $x=y$ or $x\neq y$ to indicate whether or not the variables $x$ and $y$ denote different people. The domain of discourse is all people. Let the predicate $H(x)$ mean that ``$x$ is happy,'' and let the predicate $L(x,y)$ mean that ``$x$ loves $y$.''

\subitem (i) At least one person is happy.
\subitem (ii) At least two people are happy.
\subitem (iii) At least one person is unhappy.
\subitem (iv) Exactly one person is happy.
\subitem (v) Not everyone loves someone else. (Write this in 3 different ways.)


\item Let the domain of discourse be all members of the class and let $L(x,y)$ be the predicate ``$x$ likes $y$." Write the following colloquial English sentences using quantificational formulas. The sentences are not necessarily unambiguous. If a sentence has more than one possible meaning, explain the ambiguity and which interpretation you have chosen.

\begin{enumerate}

\item Everyone in the class likes some other member of the class.

\item Someone doesn't like anybody and nobody likes that person.

\item At least three different people like the same person.

\end{enumerate}


\item A certain cabal (\emph{cabal: a secret political clique or faction})
 within the CS~20 course staff is
plotting to make the final exam \emph{ridiculously hard}.
(``Problem~1.  Prove the Poincare Conjecture starting from the axioms
of ZFC.  Express your answer in khipu---the knot language of the
Incas.'')  The only way to stop their evil plan is to determine
exactly who is in the cabal.  The course staff includes the following six
people: \[
\PersonA,\PersonB,\PersonC,\PersonD,\PersonE,\PersonF \]
The cabal is a subset of these six.  A membership roster has been
found and appears below, but it is deviously encrypted in logic
notation.  The predicate $\cabal$ indicates who is in the cabal; that is,
$\cabal(x)$ is true if and only if $x$ is a member of the cabal.  Use the information below 
to deduce who is in the cabal. (Credit: adapted from Albert R. Meyer / MIT 6.042)
\begin{gather}
\exists x\exists y\exists z \,
    (x \neq y \wedge
     x \neq z \wedge
     y \neq z \wedge
     \cabal(x) \wedge \cabal(y) \wedge \cabal(z))
\\
\cabal(\PersonF) \Rightarrow \forall x\, \cabal(x) \\ \lnot(\cabal(\PersonA) \wedge \cabal(\PersonD)) \\
\cabal(\PersonD) \Rightarrow \cabal(\PersonA)\\
(\cabal(\PersonE) \vee \cabal(\PersonC)) \Rightarrow \lnot(\cabal(\PersonB))
\end{gather}


\item The domain of discourse is the set of all finite-length binary strings as in Meyer, problem 3.21. The predicates Substring($x,y$) (meaning $x$ is a substring of $y$) and Prefix($x,y$) (meaning $x$ is a prefix of $y$) are available. Note that your answers are not limited to the use of the Substring and Prefix predicates.

\begin{enumerate}
\item Write a predicate Suffix($x, y$) that means $x$ is a suffix of $y$.


\item Write an expression that means $x$ consists of alternating 0s and 1s, e.g 01010 or 101010. (Hint: Consider substrings of size two.)


\item Write an expression that means $x$ is the binary representation of an integer of the form $5(2^k)$ for some integer $k \geq 0$.
\end{enumerate}

\end{enumerate}
\end{document}
