\documentclass[solution, letterpaper]{cs20inclass}
\usepackage{enumerate}
\usepackage{tikz}
\usepackage{pgf}
\usepackage{tikz}
\usepackage{hyperref}
\begin{document}
\header{8}{Wednesday, February 10, 2016}

\noindent Author: Michelle Danoff% \\

\paragraph*{Executive Summary}

\begin{enumerate}

\item Propositions and predicates
\begin{itemize}

\item A proposition $P$ is like a boolean variable. Its value is either ``true'' or ``false.''
\item A predicate $P(x)$ is like a boolean-valued function. It may have the value ``true'' for some some set of input values and the value ``false'' for others.
\item A predicate can have more than one argument; e.g. Enrolled($x, y$) might mean ``student $x$ is enrolled in course $y$.'' 
\item The values that  $x$ and $y$ might assume are the domain of the function. 
\end{itemize}

\item Quantifiers
\begin{itemize}
\item The existential quantifier: $\exists x P(x)$ or $\exists x. P(x)$ or $\exists x \mbox{ s.t. }P(x)$ means there exists at least one $x$ in the domain $D$ such that $P(x)$ is true.

\item The universal quantifier: $\forall x P(x)$ or $\forall x. P(x)$  means for every $x$ in the domain $D$, $P(x)$ is true.

\end{itemize}

\item Multiple quantifiers
\begin{itemize}

\item  $\exists x P(x)$ and $\forall x P(x)$ are both propositions, subject to the rules of logic that you already know.

\item  $\exists x P(x,y)$ and $\forall x P(x,y)$ are both predicates of the form $Q(y)$, subject to the rules of quantificational logic that you are learning.

\item $\exists x \exists y P(x,y)$ and $\forall x \forall y P(x,y)$ are both propositions. The order of the quanitifiers is irrelevant.

\item $\exists x \forall y P(x,y)$ and $\forall y \exists x P(x,y)$ are both propositions, but they are not equivalent! The order of the quantifiers is important. 

\end{itemize}

\item Negation and quantifiers
\begin{itemize}
\item $\neg (\exists x. P(x)) \leftrightarrow \forall x. (\neg P(x))$.
\item $\neg (\forall x. P(x)) \leftrightarrow \exists x. (\neg P(x))$.

\end{itemize}


\pagebreak
\end{enumerate}

\problem Convert the following English sentences into logical formulas. You may use expressions like $x=y$ or $x\neq y$ to indicate whether or not the variables $x$ and $y$ denote different people. The domain of discourse is all people. Let the predicate $H(x)$ mean that ``$x$ is happy,'' and let the predicate $L(x,y)$ mean that ``$x$ loves $y$.''

\subproblem At least one person is happy.
\subproblem At least two people are happy.
\subproblem At least one person is unhappy.
\subproblem Exactly one person is happy.
\subproblem Not everyone loves someone else. (Write this in 3 different ways.)

\begin{solution}
\subsolution $\exists x. H(x)$
\subsolution $\exists x.y.H(x) \wedge H(y)$
\subsolution $\exists x. \lnot H(x)$
\subsolution $\exists x. H(x) \wedge \forall y (H(y) \rightarrow x = y)$
\subsolution $\exists x. \forall y(\lnot L(x,y)), \lnot(\forall x \exists y. L(x,y)), \exists x . \lnot (\exists y . L(x,y))$

\end{solution}

\problem Let the domain of discourse be all members of the class and let $L(x,y)$ be the predicate ``$x$ likes $y$." Write the following colloquial English sentences using quantificational formulas. The sentences are not necessarily unambiguous. If a sentence has more than one possible meaning, explain the ambiguity and which interpretation you have chosen.

\subproblem Everyone in the class likes some other member of the class.
\subproblem Someone doesn't like anybody and nobody likes that person.
\subproblem At least three different people like the same person.

\begin{solution}

\subsolution $\forall x \exists y . (L(x,y) \wedge x \neq y)$
\subsolution $\exists x \forall y . (\lnot L(x,y) \wedge \lnot L(y,x))$
\subsolution $\exists x. y. z. w. (L(x,w) \land L(y,w) \land L(z,w) \land x \neq y \neq z)$

\end{solution}

\problem A certain cabal (\emph{cabal: a secret political clique or faction})
 within the CS~20 course staff is
plotting to make the final exam \emph{ridiculously hard}. The only way to stop their evil plan is to determine exactly who is in the cabal. Some of the course staff include: Michelle, Erin, Tom, Hannah, Crystal, and Jack. The cabal is a subset of these six. A membership roster ahs been found and appears below, but it is deviously encrypted in logic notation. The predicate $cabal$ indicates who is in the cabal; that is, $cabal(x)$ is true if and only if $x$ is a member of the cabal. Use the following information to gather who is in the cabal. 

\subproblem $\exists x \exists y \exists z (x \neq y \land x \neq z \land y \neq z \land cabal(x) \land cabal(y) \land cabal(z))$
\subproblem $cabal(Jack) \rightarrow \forall x cabal(x)$
\subproblem $\lnot (cabal(Michelle) \land cabal(Hannah))$
\subproblem $cabal(Hannah) \rightarrow cabal(Michelle)$
\subproblem $cabal(Crystal) \land cabal(Tom) \rightarrow \lnot(cabal(Erin))$

\begin{solution}
Michelle, Crystal, and Tom are in the cabal
\end{solution}

\problem Let us consider all finite length binary strings. We may use the following operations: $substring(x,y)$, which will return true if x is a substring of y, and $prefix(x,y)$, which will return true if x is a prefix of y. For the following problems, assume input $x$ consists of exclusively 0s and 1s. 

\subproblem Write an expression that means $x$ consists of alternating 0s and 1s. For example, 01010101 or 10101010, but not 110010100. 

\subproblem Write an expression that means that $x$ is the binary representation of an integer of the form $5(2^k)$ for some integer $k \geq 0$. 

\begin{solution}
\subsolution $\lnot substring(11, x) \land \lnot substring(00,x)$
\subsolution $ prefix(101, x) \land \lnot substring(11, x) \land \lnot substring(10101, x) \land \lnot substring(001,x)$

\end{solution}

\end{document}
