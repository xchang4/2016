\documentclass[solution, letterpaper]{cs20inclass}
\usepackage{enumerate}
\usepackage{tikz}
\usepackage{pgf}
\usepackage{tikz}
\usepackage{hyperref}
\begin{document}
\header{7}{Monday, February 8, 2016}

\noindent Author: Jack Dent% \\

\paragraph*{Executive Summary}
\begin{enumerate}

\item Forms of propositional formulas

\begin{itemize}
  \item  Disjunctive normal form: \textsc{Or}s of \textsc{And}-terms, where each \textsc{And}-term consists of variables (or negations of variables), e.g. $(p \land q) \lor (p \land r) \lor (q \land \lnot r \land t) $, or $(p \land q \land  r) \lor (\lnot p \land q \land r)\lor (\lnot p \land \lnot q \land r) $.
  \item Conjunctive form: \textsc{And}s of \textsc{Or}-terms, where each \textsc{Or}-term consists of variables (or negations of variables), e.g. $(p \lor \lnot q) \land (p \lor r) \land (q \lor \lnot r \lor t)$, or $(p \lor q \lor r) \land (p \lor \lnot q \lor r) \land (\lnot p \lor q \lor r)$.

\end{itemize}

\item De Morgan's Laws and Double Negation:
  \begin{itemize}
    \item $\lnot (A \land B) \longleftrightarrow \lnot A \lor \lnot B$
    \item $\lnot (A \lor B) \longleftrightarrow \lnot A \land \lnot B$
    \item $\lnot (\lnot A) \longleftrightarrow A$
  \end{itemize}

\item Associative Laws
  \begin{itemize}
    \item $(A \land B) \land C \longleftrightarrow A \land (B \land C) \longleftrightarrow A \land B \land C$
    \item $(A \lor B) \lor C \longleftrightarrow A \lor (B \lor C) \longleftrightarrow A \lor B \lor C$
  \end{itemize}

\item Distributive Laws
  \begin{itemize}
    \item $A \land (B \lor C) \longleftrightarrow (A \land B) \lor (A \land C)$
     \item $A \lor (B \land C) \longleftrightarrow (A \lor B) \land (A \lor C)$
  \end{itemize}

\item Tautologies and Satisfiability
  \begin{itemize}
    \item Tautology:  Formula that is true under all possible truth assignments.
    \item Satisfiable:  Formula that is true for at least one truth assignment.
    \item Unsatisfiable:  Formula whose negation is a tautology.
  \end{itemize}

\end{enumerate}
\pagebreak

\problem

\subproblem Explain why $A \rightarrow B \equiv \lnot A \lor B$ (Hint: use truth tables in your answer).

\subproblem The following formula is satisfiable.  \emph{Without using a truth table}, find a satisfying assignment, and explain your method:
	$$(((p \lor q \lor r) \land (\neg r \land s)) \lor \neg p) \rightarrow ((\neg s \lor t) \oplus \neg q) $$

\begin{solution}

\subsolution
  First, let's consider the truth table for $A \rightarrow B$. If $A$ is true, then $B$ must also be true. If $A$ isn't true, then we do not require that $B$ be either true or false.
  \begin{center}
    \begin{tabular}{|c|c|c|}\hline
      $A$ & $B$ & $A \rightarrow B$ \\\hline
      0 & 0 & 1 \\
      0 & 1 & 1 \\
      1 & 0 & 0 \\
      1 & 1 & 1 \\\hline
    \end{tabular}
  \end{center}
  
  Next, let's write out the truth table for $\lnot A \lor B$.
  \begin{center}
    \begin{tabular}{|c|c|c|}\hline
      $A$ & $B$ & $\lnot A \lor B$ \\\hline
      0 & 0 & 1 \\
      0 & 1 & 1 \\
      1 & 0 & 0 \\
      1 & 1 & 1 \\\hline
    \end{tabular}
  \end{center}
  
  Since the truth tables are identical, we can conclude that the two expressions are equivalent.

\subsolution Rewrite the formula as $A \rightarrow B \equiv \lnot A \lor B$, by the equivalence in Problem 1. If we set $p$ to 1 and $s$ to 0, then $A$ will evaluate to 0 and we have found a satisfying assignment. We can then populate the other variables with whatever values we wish. The following assignment is satisfying: $p = 1$, $q = 0$, $r = 0$, $s = 0$, $t = 0$.

\end{solution}

\problem

\subproblem Put the following formula in conjunctive normal form: $(\lnot (p \oplus q)) \rightarrow r$.
\subproblem Put the following formula in disjunctive normal form: $p \lor (q \rightarrow p)$

\begin{solution}
\subsolution We should remove the $\rightarrow$ immediately, and then remove the double negation, which leaves $(p \oplus q) \lor r$. This expands to $((p \lor q) \land \lnot (p \land q)) \lor r \equiv ((p \lor q) \land (\lnot p \lor \lnot q)) \lor r$, using De Morgan's Laws. Finally, we can use the distributive property of \textsc{Or} to reduce the expression to $(p \lor q \lor r) \land (\lnot p \lor \lnot q \lor r)$, which is in CNF.
\subsolution $p \lor (q \rightarrow p) \equiv p \lor (\lnot q \lor p) \equiv p \lor \lnot q$. This is in conjunctive and disjunctive form, since it only has one clause!
\end{solution}

\problem

\subproblem Find an easy way to check whether a formula in disjunctive normal form is satisfiable.
\subproblem (BONUS) Sam wakes up one night thinking that he has solved the P=?NP question:  P = NP!  Given a logical formula, just put the formula into disjuntive normal form, then use the method in part (a) to check to see if it is satisfiable. Why won't Sam win the million dollar prize for settling the question?

\begin{solution}
\subsolution A DNF formula is satisfiable if and only if at least one of its terms is satisfiable, and a term is satisfiable if and only if it does not contain both $p$ and $\lnot p$ for some variable $p$. This can be checked in polynomial time.
\subsolution It can take exponential time to convert a formula to disjunctive normal form.
\end{solution}

\problem

(BONUS) Determine which of the following are equivalent to $(p \land q) \rightarrow r$ and which are equivalent to $(p \lor q) \rightarrow r$:

\subproblem $p \rightarrow (q \rightarrow r)$
\subproblem $q \rightarrow (p \rightarrow r)$
\subproblem $(p \rightarrow r) \land (q \rightarrow r)$
\subproblem $(p \rightarrow r) \lor (q \rightarrow r)$

\begin{solution}
\subsolution Equivalent to $(p \land q) \rightarrow r$
\subsolution Equivalent to $(p \land q) \rightarrow r$
\subsolution Equivalent to $(p \lor q) \rightarrow r$
\subsolution Equivalent to $(p \land q) \rightarrow r$
\end{solution}

\end{document}
