%%%%%%%%%%%%%%%%%%%%%%%%%%%%%%%%%%%%%%%%%%%%%%%
\documentclass[solution, letterpaper]{cs20exam}
\usepackage{enumerate}
\usepackage{tikz}
\usepackage{pgf}
\usepackage{tikz}
\usepackage{hyperref}

\newcommand{\namebox}{\framebox(250,20){}}

\begin{document}
\header{2}{Wednesday, March 23, 2016}

\vspace{-1.8in}\hbox{}\hspace{2in}Your Name: \namebox
\vspace{1.2in}

%% Written by Tom %%

\problem{}{}

For each of the following, state whether the set is finite, countably infinite, or uncountable. No justification required.

\subproblem The set of all total functions with domain $\{0, 1\}$ and codomain $\{0, 1\}$.
\subproblem The set of all total functions with domain $\mathbb{N}$ and codomain $\{1\}$.
\subproblem The set of all total functions with domain $\mathbb{N}$ and codomain $\{0, 1\}$.
\subproblem The set of all total functions with domain $\{0, 1\}$ and codomain $\mathbb{N}$.

\begin{solution}
\subsolution Finite
\subsolution Finite
\subsolution Uncountable
\subsolution Countably Infinite
\end{solution}

%% End of problems written by Tom %%

%% Written by Ben %%

\problem{}{}

Draw state machines that only accept strings in the following set. Assume that the alphabet is $\Sigma = \{0, 1\}$; that is, for all possible input strings $s$ we have $s \in \Sigma^*$.

$$\{w : w \text{ starts with } 0 \text{ and contains the substring } 101 \text{, i.e. } w = 0x101y \text{ for some } x \text{ and } y \}$$

\begin{solution}
\includegraphics[width=15cm]{midterm2statesolution}
\end{solution}

%% End of problems written by Ben %%

%% Written by Jack %%

\problem{}{}

Let $G$ be a directed graph with $n$ vertices. Show that if $G$ has a path of length greater than $n$, then $G$ has a cycle (a path has length $k$ if it contains $k$ edges).

\begin{solution}

Proof by contradiction: assume there is a path of length $m > n$. The path can be written as: $(v_0, v_1), (v_1, v_2), (v_2, v_3), \ldots, (v_{m-1}, v_m)$. There are $m + 1 > n + 1$ vertices on the path. By the pigeonhole principle, at least 2 of those vertices are equal as the graph only has n vertices. Let the index of the first instance of the vertex in the path be $i$ and the index of the second instance of the vertex be $j$, then there is a cycle starting and ending at that vertex with the path $(v_i, v_{i+1}), (v_{i+1}, v_{i+2}), \ldots, (v_{j-1}, v_{j})$.

\end{solution}

\problem{}{}
Let $G = (V, E)$ be a directed acyclic graph. Define a relation $R$ on $V$ by $(v_1, v_2)$ which is an element of $R$ iff there is a path from $v_1$ to $v_2$.

\subproblem Is $R$ reflexive? Prove your answer.
\subproblem Is $R$ symmetric? Prove your answer.
\subproblem Is $R$ transitive? Prove your answer.

\begin{solution}

\subsolution $R$ is not reflexive. A DAG contains no cycles, and therefore there cannot be any paths from a vertex to itself.

\subsolution $R$ is not symmetric. Proof by contradiction: suppose $R$ is symmetric, and that there is a path from $(v_1, v_2) \in R$. Then there must be a path from $v_2$ to $v_1$, which implies that $(v_1, v_1) \in R$. We have already shown that this cannot be the case in a DAG, so $R$ cannot be symmetric.

\subsolution $R$ is transitive. If there is a path from $v_1$ to $v_2$ and a path from $v_2$ to $v_3$, we can concatenate the two paths to construct a path from $v_1$ to $v_3$.

\end{solution}

%% End of problems written by Jack %%

\pagebreak

%% Written by Hannah %%

\problem{}{}

\subproblem Using set-builder notation, give a formal description of the union of two sets $A$ and $B$. % Tests union
\subproblem Using set-builder notation, give a formal description of the complement of a set $A$. % Tests difference
\subproblem Let $|A| = n$ and $|B| = m$. If $A \subseteq B$, what is $|A \cap B|$? % Tests intersection
\subproblem Let $|A| = n$ and $|B| = m$. If $A \subseteq B$, what is $|A - B|$? % Tests difference
\subproblem What is the power set of $\{ h, a, i\}$? % Tests power set

\begin{solution}
\subsolution $\{x : x \in A \vee x \in B\}$
\subsolution $\{x : x \notin A\}$
\subsolution $n$
\subsolution 0
\subsolution $\{\emptyset, \{h\}, \{a\}, \{i\}, \{h, a\}, \{h, i\}, \{a, i\}, \{h, a, i\} \}$
\end{solution}

%% Written by Crystal %%

\problem{}{}

Let S be the set defined as follows:
\begin{itemize}
\item Base Case: $(1,2) \in S$
\item Constructor Rules: If $(x,y)\in S$, then C1: $(x+2, y) \in S$ and C2: $(y,x) \in S$ 
\end{itemize}
Use structural induction to prove that for any pair (x,y) in S, x and y can not both be odd or both be even. 

\begin{solution}
\begin{itemize}
\item Basis step: By the base case of the definition of S, $(1,2)\in S$. 1 is even and 0 is odd.
\item Recursive step: \\
Assume elements (x,y) in S and P(x,y): \textit{x and y are not both be odd or both be even} holds. \\
Now consider the constructor rule in the definition of S.  We want to prove P(x+2,y) and P(y,x) also holds.
\subitem Case 1:  x is even and y is odd,
		\subsubitem x+2 would be even, y would be odd $\rightarrow$ P(x+2,y) holds.
		\subsubitem y would be odd, x would be even $\rightarrow$ P(y,x) holds.
\subitem Case 2:  x is odd and y is even,
		\subsubitem x+2 would be odd, y would be even $\rightarrow$ P(x+2,y) holds.
		\subsubitem y would be even, x would be odd $\rightarrow$ P(y,x) holds.
\end{itemize}
\end{solution}

\end{document}
