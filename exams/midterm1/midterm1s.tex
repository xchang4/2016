%%%%%%%%%%%%%%%%%%%%%%%%%%%%%%%%%%%%%%%%%%%%%%%
\documentclass[solution, letterpaper]{cs20exam}
\usepackage{enumerate}
\usepackage{tikz}
\usepackage{pgf}
\usepackage{tikz}
\usepackage{hyperref}
\begin{document}
\header{1}{Monday, February 22, 2016}

\problem{}{} Prove that if you pick $5$ distinct integers from $\{1, \ldots, 100\}$, some two differ by at most $24$.

\begin{solution}
Divide the integers $\{1, \ldots, 100\}$ into the 4 sets of integers $\{1, \ldots, 25\}$, $\{26, \ldots, 50\}$, $\{51, \ldots, 75\}$, and $\{76, \ldots, 100\}$. Let each of this four sets be a pigeonhole. We are picking 5 integers from $\{1, \ldots, 100\}$. Let each of these five integers be a pigeon. By the pigeonhole principle there must be two integers in the same set.  

Since the smallest and largest integers in each set differ by only 24, the 2 integers from the same set can differ by at most 24, and so there must be some 2 of our 5 integers that differ by at most 24.
\end{solution}

\problem{}{} Let $A = \neg p \lor q$ and $B = p \oplus \neg q$. Show by writing out the two truth tables that these two formulas are not equivalent.

\begin{solution}
\begin{table}[h]
\centering
\label{my-label}
\begin{tabular}{llll}
$p$ & $q$ & $\neg p \lor q$ & $p \oplus \neg q$ \\
1   & 1   & 1               & 1                 \\
1   & 0   & 0               & 0                 \\
0   & 1   & 1               & 0                 \\
0   & 0   & 1               & 1                
\end{tabular}
\end{table}
\end{solution}

\problem{}{} Perform the following operations in binary. Show your work.

\subproblem $1111_2 + 1111_2$
\subproblem $1010_2 - 100_2$

\begin{solution}
\subsolution $11110_2$
\subsolution $110_2$
\end{solution}

\problem{}{} Let the domain of discourse be all Harvard CS courses. The predicate $P(c, d)$ means that course $c$ is a prerequisite for course $d$. Assume that it is not possible for a course to be a prerequisite of itself. Write the following English sentences using quantificational formulas.

\subproblem There is a course that is a prerequisite for every other course.
\subproblem At least one course is a prerequisite for exactly one other course.

\begin{solution}
\subsolution $\exists c \forall d: (d \neq c) \implies P(c,d)$
\subsolution $\exists c, a . \forall b : (a \neq c) \land P(c, a) \land (P(c, b) \implies a = b)$

\end{solution}

\problem{}{}
Let $a_1=1$ and for every $k \geq 1$, $a_{k+1}=3a_k+1$. Prove that for every $k \geq1 $, $a_k$ is odd or even depending on whether $k$ is odd or event, respectively. You may assume that the product of an even number and another number is even, and the product of two odd numbers is odd.

\begin{solution}
Let $P(n)$ be the predicate that $a_n$ is odd if and only if $n$ is odd. Base case: $P(1)$ holds, since $1 = a_1$ are both odd. Inductive hypothesis: suppose that $P(n)$ holds for $n \ge 1$. Case 1: $n$ is even. Then $n+1$ is odd, and $a_{n+1} = 3a_n + 1$. By the inductive hypothesis, $a_n$ must be even. An even number times an even number is always even, so $3a_n$ is even. An even number $+1$ is odd, so $a_{n+1}$ is odd. Case 2: $n$ is odd. By the inductive hypothesis, $a_n$ must be odd. An odd number times an odd number is odd, so $3a_n$ is odd. An odd number $+1$ is even, so $a_{n+1}$ is even.
\end{solution}

\problem{}{}
Prove that in any group of six people, at least two of them know the same number of people. (You don't know yourself, and if you know someone, then that person knows you too.)

\begin{solution}

There are 6 possible numbers of people a person can know (0 through 5 since you cannot know yourself).

Suppose that each person knows a different number of people. Then someone (person A) knows 0 people and someone (person B) else knows 5 people. Since B knows 5 people and cannot know him or herself, then B knows A. Since knowing is a symmetric relation, A knows B. This is a contradiction because A knows 0 people.

Since we have arrived at a contradiction, we can conclude that in any group of six people, at least two of them know the same number of people.

\end{solution}


\end{document}
